\documentclass[a4paper,11pt,oneside]{memoir}

% Castellano
\usepackage[spanish,es-tabla]{babel}
\selectlanguage{spanish}
\usepackage[utf8]{inputenc}
\usepackage{placeins}
\usepackage{float}

\RequirePackage{booktabs}
\RequirePackage[table]{xcolor}
\RequirePackage{xtab}
\RequirePackage{multirow}

% Multi-page tables using
\usepackage{longtable}
\usepackage{tabularx}

% Cell with line break (e.g. \specialcell{Foo\\bar})
\newcommand{\specialcell}[2][c]{%
  \begin{tabular}[#1]{@{}l@{}}#2\end{tabular}}

% Mathematic font
\usepackage{amsfonts}

% Color

% Bibliography management
\usepackage[numbers,sort]{natbib}

% Links
\usepackage[colorlinks]{hyperref}
\hypersetup{
	colorlinks,
	linkcolor={green!40!black},
	citecolor={blue!50!black},
	urlcolor={blue!80!black}
}

% Ecuaciones
\usepackage{amsmath}

% Rutas de fichero / paquete
\newcommand{\ruta}[1]{{\sffamily #1}}

% Párrafos
\nonzeroparskip

% Listas estrechas
\providecommand{\tightlist}{%
  \setlength{\itemsep}{0pt}\setlength{\parskip}{0pt}}

% Imagenes
\usepackage{graphicx}
\newcommand{\imagen}[2]{
	\begin{figure}[!h]
		\centering
		\includegraphics[width=0.9\textwidth]{#1}
		\caption{#2}\label{fig:#1}
	\end{figure}
	\FloatBarrier
}

\newcommand{\imagenflotante}[2]{
	\begin{figure}%[!h]
		\centering
		\includegraphics[width=0.9\textwidth]{#1}
		\caption{#2}\label{fig:#1}
	\end{figure}
}



% El comando \figura nos permite insertar figuras comodamente, y utilizando
% siempre el mismo formato. Los parametros son:
% 1 -> Porcentaje del ancho de página que ocupará la figura (de 0 a 1)
% 2 --> Fichero de la imagen
% 3 --> Texto a pie de imagen
% 4 --> Etiqueta (label) para referencias
% 5 --> Opciones que queramos pasarle al \includegraphics
% 6 --> Opciones de posicionamiento a pasarle a \begin{figure}
\newcommand{\figuraConPosicion}[6]{%
  \setlength{\anchoFloat}{#1\textwidth}%
  \addtolength{\anchoFloat}{-4\fboxsep}%
  \setlength{\anchoFigura}{\anchoFloat}%
  \begin{figure}[#6]
    \begin{center}%
      \Ovalbox{%
        \begin{minipage}{\anchoFloat}%
          \begin{center}%
            \includegraphics[width=\anchoFigura,#5]{#2}%
            \caption{#3}%
            \label{#4}%
          \end{center}%
        \end{minipage}
      }%
    \end{center}%
  \end{figure}%
}

%
% Comando para incluir imágenes en formato apaisado (sin marco).
\newcommand{\figuraApaisadaSinMarco}[5]{%
  \begin{figure}%
    \begin{center}%
    \includegraphics[angle=90,height=#1\textheight,#5]{#2}%
    \caption{#3}%
    \label{#4}%
    \end{center}%
  \end{figure}%
}
% Para las tablas
\newcommand{\otoprule}{\midrule [\heavyrulewidth]}
%
% Nuevo comando para tablas pequeñas (menos de una página).
\newcommand{\tablaSmall}[5]{%
 \begin{table}[H]
  \begin{center}
   \rowcolors {2}{gray!35}{}
   \begin{tabular}{#2}
    \toprule
    #4
    \otoprule
    #5
    \bottomrule
   \end{tabular}
   \caption{#1}
   \label{tabla:#3}
  \end{center}
 \end{table}
}

%
% Nuevo comando para tablas pequeñas (menos de una página).
\newcommand{\tablaSmallSinColores}[5]{%
 \begin{table}[H]
  \begin{center}
   \begin{tabular}{#2}
    \toprule
    #4
    \otoprule
    #5
    \bottomrule
   \end{tabular}
   \caption{#1}
   \label{tabla:#3}
  \end{center}
 \end{table}
}

\newcommand{\tablaApaisadaSmall}[5]{%
\begin{landscape}
  \begin{table}
   \begin{center}
    \rowcolors {2}{gray!35}{}
    \begin{tabular}{#2}
     \toprule
     #4
     \otoprule
     #5
     \bottomrule
    \end{tabular}
    \caption{#1}
    \label{tabla:#3}
   \end{center}
  \end{table}
\end{landscape}
}

%
% Nuevo comando para tablas grandes con cabecera y filas alternas coloreadas en gris.
\newcommand{\tabla}[6]{%
  \begin{center}
    \tablefirsthead{
      \toprule
      #5
      \otoprule
    }
    \tablehead{
      \multicolumn{#3}{l}{\small\sl continúa desde la página anterior}\\
      \toprule
      #5
      \otoprule
    }
    \tabletail{
      \hline
      \multicolumn{#3}{r}{\small\sl continúa en la página siguiente}\\
    }
    \tablelasttail{
      \hline
    }
    \bottomcaption{#1}
    \rowcolors {2}{gray!35}{}
    \begin{xtabular}{#2}
      #6
      \bottomrule
    \end{xtabular}
    \label{tabla:#4}
  \end{center}
}

%
% Nuevo comando para tablas grandes con cabecera.
\newcommand{\tablaSinColores}[6]{%
  \begin{center}
    \tablefirsthead{
      \toprule
      #5
      \otoprule
    }
    \tablehead{
      \multicolumn{#3}{l}{\small\sl continúa desde la página anterior}\\
      \toprule
      #5
      \otoprule
    }
    \tabletail{
      \hline
      \multicolumn{#3}{r}{\small\sl continúa en la página siguiente}\\
    }
    \tablelasttail{
      \hline
    }
    \bottomcaption{#1}
    \begin{xtabular}{#2}
      #6
      \bottomrule
    \end{xtabular}
    \label{tabla:#4}
  \end{center}
}

%
% Nuevo comando para tablas grandes sin cabecera.
\newcommand{\tablaSinCabecera}[5]{%
  \begin{center}
    \tablefirsthead{
      \toprule
    }
    \tablehead{
      \multicolumn{#3}{l}{\small\sl continúa desde la página anterior}\\
      \hline
    }
    \tabletail{
      \hline
      \multicolumn{#3}{r}{\small\sl continúa en la página siguiente}\\
    }
    \tablelasttail{
      \hline
    }
    \bottomcaption{#1}
  \begin{xtabular}{#2}
    #5
   \bottomrule
  \end{xtabular}
  \label{tabla:#4}
  \end{center}
}



\definecolor{cgoLight}{HTML}{EEEEEE}
\definecolor{cgoExtralight}{HTML}{FFFFFF}

%
% Nuevo comando para tablas grandes sin cabecera.
\newcommand{\tablaSinCabeceraConBandas}[5]{%
  \begin{center}
    \tablefirsthead{
      \toprule
    }
    \tablehead{
      \multicolumn{#3}{l}{\small\sl continúa desde la página anterior}\\
      \hline
    }
    \tabletail{
      \hline
      \multicolumn{#3}{r}{\small\sl continúa en la página siguiente}\\
    }
    \tablelasttail{
      \hline
    }
    \bottomcaption{#1}
    \rowcolors[]{1}{cgoExtralight}{cgoLight}

  \begin{xtabular}{#2}
    #5
   \bottomrule
  \end{xtabular}
  \label{tabla:#4}
  \end{center}
}


















\graphicspath{ {../img/} }

% Capítulos
\chapterstyle{bianchi}
\newcommand{\capitulo}[2]{
	\setcounter{chapter}{#1}
	\setcounter{section}{0}
	\chapter*{#2}
	\addcontentsline{toc}{chapter}{#2}
	\markboth{#2}{#2}
}

% Apéndices
\renewcommand{\appendixname}{Apéndice}
\renewcommand*\cftappendixname{\appendixname}

\newcommand{\apendice}[1]{
	%\renewcommand{\thechapter}{A}
	\chapter{#1}
}

\renewcommand*\cftappendixname{\appendixname\ }

% Formato de portada
\makeatletter
\usepackage{xcolor}
\newcommand{\tutor}[1]{\def\@tutor{#1}}
\newcommand{\course}[1]{\def\@course{#1}}
\definecolor{cpardoBox}{HTML}{E6E6FF}
\def\maketitle{
  \null
  \thispagestyle{empty}
  % Cabecera ----------------
\noindent\includegraphics[width=\textwidth]{cabecera}\vspace{1cm}%
  \vfill
  % Título proyecto y escudo informática ----------------
  \colorbox{cpardoBox}{%
    \begin{minipage}{.8\textwidth}
      \vspace{.5cm}\Large
      \begin{center}
      \textbf{TFG del Grado en Ingeniería Informática}\vspace{.6cm}\\
      \textbf{\LARGE\@title{}}
      \end{center}
      \vspace{.2cm}
    \end{minipage}

  }%
  \hfill\begin{minipage}{.20\textwidth}
    \includegraphics[width=\textwidth]{escudoInfor}
  \end{minipage}
  \vfill
  % Datos de alumno, curso y tutores ------------------
  \begin{center}%
  {%
    \noindent\LARGE
    Presentado por \@author{}\\ 
    en la Universidad de Burgos --- \@date{}\\
    Tutores: \@tutor{}\\
  }%
  \end{center}%
  \null
  \cleardoublepage
  }
\makeatother

\newcommand{\nombre}{David Miguel Lozano} %%% cambio de comando

% Datos de portada
\title{{\Huge GoBees}\\[0.5cm]Monitorización del estado de una colmena mediante la cámara de un smartphone.}
\author{\nombre}
\tutor{Dr. José Francisco Díez Pastor\\y Dr. Raúl Marticorena Sánchez}
\date{\today}

\begin{document}

\maketitle



\newpage\null\thispagestyle{empty}\newpage


%%%%%%%%%%%%%%%%%%%%%%%%%%%%%%%%%%%%%%%%%%%%%%%%%%%%%%%%%%%%%%%%%%%%%%%%%%%%%%%%%%%%%%%%
\thispagestyle{empty}


\noindent\includegraphics[width=\textwidth]{cabecera}\vspace{1cm}

\noindent D. José Francisco Díez Pastor y D. Raúl Marticorena Sánchez, profesores del departamento de Ingeniería Civil, área de Lenguajes y Sistemas Informáticos.

\noindent Exponen:

\noindent Que el alumno D. \nombre, con DNI 71307412Y, ha realizado el Trabajo Final de Grado en Ingeniería Informática titulado ``GoBees - Monitorización del estado de una colmena mediante la cámara de un smartphone''. 

\noindent Y que dicho trabajo ha sido realizado por el alumno bajo la dirección del que suscribe, en virtud de lo cual se autoriza su presentación y defensa.

\begin{center} %\large
En Burgos, {\large \today}
\end{center}

\vfill\vfill\vfill

% Author and supervisor
\begin{minipage}{0.45\textwidth}
\begin{flushleft} %\large
Vº. Bº. del Tutor:\\[2cm]
D. José Francisco Díez Pastor
\end{flushleft}
\end{minipage}
\hfill
\begin{minipage}{0.45\textwidth}
\begin{flushleft} %\large
Vº. Bº. del Tutor:\\[2cm]
D. Raúl Marticorena Sánchez
\end{flushleft}
\end{minipage}
\hfill

\vfill

% para casos con solo un tutor comentar lo anterior
% y descomentar lo siguiente
%Vº. Bº. del Tutor:\\[2cm]
%D. nombre tutor


\newpage\null\thispagestyle{empty}\newpage




\frontmatter

% Abstract en castellano
\renewcommand*\abstractname{Resumen}
\begin{abstract}
La actividad de vuelo de una colmena es un indicador importante sobre su
estado de salud. Sin embargo, la monitorización manual de este parámetro
es un proceso muy costoso y puede introducir una tasa de error elevada.
Por ello, se han desarrollado diversos métodos que permiten automatizar
este proceso.

En este trabajo se propone un nuevo método más accesible al público
general que permite la monitorización de la actividad de vuelo de una
colmena mediante la cámara de un \emph{smartphone} Android.

Además, se ha desarrollado una aplicación de gestión de colmenares que
integra el algoritmo y proporciona las herramientas necesarias para
interpretar y organizar toda la información recabada.

GoBees es la aplicación resultante y se encuentra disponible a través de
Google Play o de la página oficial del proyecto \url{http://gobees.io/}.
\end{abstract}

\renewcommand*\abstractname{Descriptores}
\begin{abstract}
Contador de abejas, actividad de vuelo, monitorización de colmenas, 
gestión de colmenares, aplicación Android.
\end{abstract}

\clearpage

% Abstract en inglés
\renewcommand*\abstractname{Abstract}
\begin{abstract}
Flight activity of a honey bee colony is an overall indicator of the
state of the hive's health. However, manually monitoring this parameter
is a very expensive and time-consuming process and can introduce a high
error rate. Thus, several methods to automate this process have been
developed over the years.

In this work, we propose a new method more accessible to the general
public which allows monitoring the flight activity of a honey bee colony
using the built-in camera of an Android smartphone.

In addition, an apiary management application has been developed,
incorporating the algorithm and providing the necessary tools to
interpret and organize all the information gathered.

GoBees is the resulting application and it is available from Google Play
or the official web site \url{http://gobees.io/}.
\end{abstract}

\renewcommand*\abstractname{Keywords}
\begin{abstract}
Bee counter, flight activity, hive monitoring, apiary management, 
Android application.
\end{abstract}

\clearpage

% Indices
\tableofcontents

\clearpage

\listoffigures

\clearpage

%\listoftables

%\clearpage

\mainmatter
\capitulo{1}{Introducción}

Las abejas son una pieza clave en nuestro ecosistema. La producción de
alimentos a nivel mundial y la biodiversidad de nuestro planeta dependen
en gran medida de ellas. Son las encargadas de polinizar el 70\% de los
cultivos de comida, que suponen un 90\% de la alimentación humana \citep{art:bees_decline}. 
Sin embargo, la población mundial de abejas
está disminuyendo a pasos agigantados en los últimos años debido, entre
otras causas, al uso extendido de plaguicidas tóxicos, parásitos como la
varroa o la expansión del avispón asiático \citep{art:ccd}.

Actualmente los apicultores inspeccionan sus colmenares de forma manual.
Uno de los indicadores que más información les proporciona es la
actividad de vuelo de la colmena (número de abejas en vuelo a la entrada
de la colmena en un determinado instante) \citep{art:campbell2008}. Este
dato, junto con información previa de la colmena y conocimiento de las
condiciones locales, permite al apicultor conocer el estado de la
colmena con bastante seguridad, pudiendo determinar si esta necesita o
no una intervención.

La actividad de vuelo de una colmena varía dependiendo de múltiples
factores, tanto externos como internos a la colmena. Entre ellos se
encuentran la propia población de la colmena, las condiciones
meteorológicas, la presencia de enfermedades, parásitos o depredadores,
la exposición a tóxicos, la presencia de fuentes de néctar, etc. A pesar
de los numerosos factores que influyen en la actividad de vuelo, su
conocimiento es de gran ayuda para la toma de decisiones por parte del
apicultor. Ya que este posee información sobre la mayoría de los
factores necesarios para su interpretación.

La captación prolongada de la actividad de vuelo de forma manual es muy
costosa, tediosa y puede introducir una tasa de error elevada. Es por
esto que a lo largo de los años se haya intentado automatizar este
proceso de muy diversas maneras. Los primeros intentos se remontan a
principios del siglo XX, donde se desarrollaron contadores por contacto
eléctrico \citep{art:lundie1925}. Otros métodos posteriores se basan en
sensores de infrarrojos \citep{art:struye1994}, sensores capacitivos
\citep{art:campbell2005}, códigos de barras \citep{beebarcode} o incluso
en microchips acoplados a las abejas
\citep{art:decourtye_honeybee_2011}. En los últimos años, se han
desarrollado numerosos métodos basados en visión artificial
\citep{art:campbell2008,art:chiron2013a,art:chiron2013,art:tashakkori2015}.

Los métodos basados en contacto, sensores de infrarrojos o capacitivos
tienen el inconveniente de que es necesario realizar modificaciones en
la colmena, mientras que en los basados en códigos de barras o
microchips es necesario manipular las abejas directamente. Estos motivos
los convierten en métodos poco prácticos fuera del campo investigador.
Por el contrario, la visión artificial aporta un gran potencial, ya que
evita tener que realizar ningún tipo de modificación ni en el entorno,
ni en las abejas. Además, abre la puerta a la monitorización de nuevos
parámetros como la detección de enjambrazón (división de la colmena y
salida de un enjambre) o la detección de amenazas (avispones, abejaruco,
etc.).

Todos los métodos basados en visión artificial propuestos hasta la fecha
utilizan hardware específico con un coste elevado. En este trabajo se
propone un método de monitorización de la actividad de vuelo de una
colmena mediante la cámara de un \emph{smartphone} con Android.

El método propuesto podría revolucionar el campo de la monitorización de
la actividad de vuelo de colmenas, ya que lo hace accesible al gran
público. Ya no es necesario contar con costoso hardware, difícil de
instalar. Solamente es necesario disponer de un \emph{smartphone} con cámara y
la aplicación GoBees. Además, esta facilita la interpretación de los
datos, representándolos gráficamente y añadiendo información adicional
como la situación meteorológica. Permitiendo a los apicultores centrar
su atención donde realmente es necesaria.

\section{Estructura de la memoria}\label{estructura-de-la-memoria}

La memoria sigue la siguiente estructura:

\begin{itemize}
\tightlist
\item
  \textbf{Introducción:} breve descripción del problema a resolver y la
  solución propuesta. Estructura de la memoria y listado de materiales
  adjuntos.
\item
  \textbf{Objetivos del proyecto:} exposición de los objetivos que
  persigue el proyecto.
\item
  \textbf{Conceptos teóricos:} breve explicación de los conceptos
  teóricos clave para la comprensión de la solución propuesta.
\item
  \textbf{Técnicas y herramientas:} listado de técnicas metodológicas y
  herramientas utilizadas para gestión y desarrollo del proyecto.
\item
  \textbf{Aspectos relevantes del desarrollo:} exposición de aspectos
  destacables que tuvieron lugar durante la realización del proyecto.
\item
  \textbf{Trabajos relacionados:} estado del arte en el campo de la
  monitorización de la actividad de vuelo de colmenas y proyectos
  relacionados.
\item
  \textbf{Conclusiones y líneas de trabajo futuras:} conclusiones
  obtenidas tras la realización del proyecto y posibilidades de mejora o
  expansión de la solución aportada.
\end{itemize}

Junto a la memoria se proporcionan los siguientes anexos:

\begin{itemize}
\tightlist
\item
  \textbf{Plan del proyecto software:} planificación temporal y estudio
  de viabilidad del proyecto.
\item
  \textbf{Especificación de requisitos del software:} se describe la
  fase de análisis; los objetivos generales, el catálogo de requisitos
  del sistema y la especificación de requisitos funcionales y no
  funcionales.
\item
  \textbf{Especificación de diseño:} se describe la fase de diseño; el
  ámbito del software, el diseño de datos, el diseño procedimental y el
  diseño arquitectónico.
\item
  \textbf{Manual del programador:} recoge los aspectos más relevantes
  relacionados con el código fuente (estructura, compilación,
  instalación, ejecución, pruebas, etc.).
\item
  \textbf{Manual de usuario:} guía de usuario para el correcto manejo de
  la aplicación.
\end{itemize}

\section{Materiales adjuntos}\label{materiales-adjuntos}

Los materiales que se adjuntan con la memoria son: 

\begin{itemize}
\tightlist
\item
	Aplicación para Android GoBees.
\item
	Aplicación Java para el desarrollo del algoritmo.
\item	
	Aplicación Java para el etiquetado de fotogramas.
\item	
	JavaDoc.
\item	
	\emph{Dataset} de vídeos de prueba.
\end{itemize}

Además, los siguientes recursos están accesibles a través de internet:

\begin{itemize}
\tightlist
\item
  Página web del proyecto \citep{gobees:web}.
\item
  GoBees en Google Play \citep{gobees:play}.
\item
  Repositorio del proyecto \citep{gobees:repo}.
\item
  Repositorio de las herramientas desarrolladas para el proyecto \citep{gobees:prototipes}.
\end{itemize}

\capitulo{2}{Objetivos del proyecto}

Este apartado explica de forma precisa y concisa cuales son los objetivos que se persiguen con la realización del proyecto. Se puede distinguir entre los objetivos marcados por los requisitos del software a construir y los objetivos de carácter técnico que plantea a la hora de llevar a la práctica el proyecto.

\capitulo{3}{Conceptos teóricos}

En aquellos proyectos que necesiten para su comprensión y desarrollo de unos conceptos teóricos de una determinada materia o de un determinado dominio de conocimiento, debe existir un apartado que sintetice dichos conceptos.

Algunos conceptos teóricos de \LaTeX \footnote{Créditos a los proyectos de Álvaro López Cantero: Configurador de Presupuestos y Roberto Izquierdo Amo: PLQuiz}.

\section{Secciones}

Las secciones se incluyen con el comando section.

\subsection{Subsecciones}

Además de secciones tenemos subsecciones.

\subsubsection{Subsubsecciones}

Y subsecciones. 

\section{Referencias}

Las referencias se incluyen en el texto usando cite \cite{wiki:latex}. Para citar webs, artículos o libros \cite{koza92}.


\section{Imágenes}

Se pueden incluir imágenes con los comandos standard de \LaTeX, pero esta plantilla dispone de comandos propios como por ejemplo el siguiente:

\imagen{thompson-vacio}{Autómata para una expresión vacía}



\section{Listas de items}

Existen tres posibilidades:

\begin{itemize}
	\item primer item.
	\item segundo item.
\end{itemize}

\begin{enumerate}
	\item primer item.
	\item segundo item.
\end{enumerate}

\begin{description}
	\item[Primer item] más información sobre el primer item.
	\item[Segundo item] más información sobre el segundo item.
\end{description}
	


\section{Tablas}

Igualmente se pueden usar los comandos específicos de \LaTeX o bien usar alguno de los comandos de la plantilla.

\tablaSmall{Herramientas y tecnologías utilizadas en cada parte del proyecto}{l c c c c}{herramientasportipodeuso}
{ \multicolumn{1}{l}{Herramientas} & App AngularJS & API REST & BD & Memoria \\}{ 
HTML5 & X & & &\\
CSS3 & X & & &\\
BOOTSTRAP & X & & &\\
JavaScript & X & & &\\
AngularJS & X & & &\\
Bower & X & & &\\
PHP & & X & &\\
Karma + Jasmine & X & & &\\
Slim framework & & X & &\\
Idiorm & & X & &\\
Composer & & X & &\\
JSON & X & X & &\\
PhpStorm & X & X & &\\
MySQL & & & X &\\
PhpMyAdmin & & & X &\\
Git + BitBucket & X & X & X & X\\
Mik\TeX{} & & & & X\\
\TeX{}Maker & & & & X\\
Astah & & & & X\\
Balsamiq Mockups & X & & &\\
VersionOne & X & X & X & X\\
} 

\capitulo{4}{Técnicas y herramientas}

\section{Metodologías}\label{metodologias}

\subsection{Scrum}\label{scrum}

Scrum es un marco de trabajo para el desarrollo de \emph{software} que se
engloba dentro de las metodologías ágiles. Aplica una estrategia de
trabajo iterativa e incremental a través de iteraciones (\emph{sprints})
y revisiones \citep{wiki:scrum}.

\subsection{\emph{Test-Driven Development} (TDD)}\label{test-driven-development-tdd}

Es una práctica de desarrollo de \emph{software} que se basa en la repetición
de un ciclo corto de desarrollo: transformar requerimientos a test,
desarrollar el código necesario para pasar los test y posteriormente
refactorizar el código. Esta práctica obliga a los desarrolladores a
analizar cuidadosamente las especificaciones antes de empezar a escribir
código, fomenta la escritura de test, la simplicidad del código y
aumenta la productividad. Como resultado se obtiene \emph{software} más seguro
y de mayor calidad \citep{wiki:tdd}.

\subsection{Gitflow}\label{gitflow}

Gitflow es un flujo de trabajo con Git que define un modelo estricto de ramas
diseñado en torno a los lanzamientos de proyecto. En la rama \emph{main}
se hospeda la última versión estable del proyecto. La rama
\emph{develop} contiene los últimos desarrollos realizados para el
siguiente lanzamiento. Por cada característica que se vaya a implementar
se crea una \emph{feature branch}. La preparación del siguiente
lanzamiento se realiza en una \emph{release branch}. Si aparece un fallo
en producción, este se soluciona en una \emph{hotfix branch} \citep{git:gitflow}.

\subsection{Técnica Pomodoro}\label{pomodoro}

La técnica Pomodoro es un método para incrementar la productividad aprovechando 
mejor el tiempo. Para aplicarla, se divide la tarea a realizar en intervalos de
25 minutos, llamados \emph{pomodoros}. Durante estos intervalos se debe evitar cualquier 
distracción que nos desvíe de la tarea. Después de cada Pomodoro se descansa 5 
minutos, menos en los múltiplos de cuatro que se realiza un descanso de 30 minutos
\citep{wiki:pomodoro}.

Se ha utilizado la aplicación \href{https://www.microsoft.com/es-es/store/p/focus-10/9nblggh5g2xh}{Focus 10} como temporizador.

\section{Patrones de diseño}\label{patrones-de-diseno}

\subsection{\emph{Model-View-Presenter} (MVP)}\label{model-view-presenter-mvp}

MVP es un patrón de arquitectura derivado del
\emph{Model--View--Controller} (MVC). Permite separar los datos internos
del modelo de una vista pasiva y enlazarlos mediante el \emph{presenter}
que maneja toda la lógica de la aplicación \citep{pattern:mvp}. 
Posee tres capas:

\begin{itemize}
\tightlist
\item
  \textit{\textbf{Model}}: almacena y proporciona los datos internos.
\item
  \textit{\textbf{View}}: maneja la visualización de los datos (del modelo).
  Propaga las acciones de usuario al \emph{presenter}.
\item
  \textit{\textbf{Presenter}}: enlaza las dos capas anteriores. Sincroniza los
  datos mostrados en la vista con los almacenados en el modelo y actúa
  ante los eventos de usuario propagados por la vista.
\end{itemize}

\imagen{mvp}{Patrón MVP.}

\subsection{Patrón repositorio}\label{patron-repositorio}

El patrón repositorio proporciona una abstracción de la implementación
del acceso a datos con el objetivo de que este sea transparente a la
lógica de negocio de la aplicación. Por ejemplo, las fuentes de datos
pueden ser una base de datos, un \emph{web service}, etc. El repositorio
media entre la capa de acceso a datos y la lógica de negocio de tal
forma que no existe ninguna dependencia entre ellas. Consiguiendo
desacoplar, mantener y testear más fácilmente el código y permitiendo la
reutilización del acceso a datos desde cualquier cliente \citep{pattern:repository}.

\imagen{repository_pattern}{Patrón Repositorio.}

\section{Control de versiones}\label{control-de-versiones}

\begin{itemize}
\tightlist
\item
  Herramientas consideradas: \href{https://git-scm.com/}{Git} y
  \href{https://subversion.apache.org/}{Subversion}.
\item
  Herramienta elegida: \href{https://git-scm.com/}{Git}.
\end{itemize}

Git es un sistema de control de versiones distribuido. A día de hoy, es
el sistema con mayor número de usuarios. La principal diferencia con
Subversion es su carácter descentralizado, que permite a cada
desarrollador tener una copia en local del repositorio completo. Git
está licenciado bajo la licencia de \emph{software} libre GNU LGPL v2.1.

\section{\emph{Hosting} del repositorio}\label{hosting-del-repositorio}

\begin{itemize}
\tightlist
\item
  Herramientas consideradas: \href{https://github.com/}{GitHub},
  \href{https://bitbucket.org/}{Bitbucket} y
  \href{https://gitlab.com/}{GitLab}.
\item
  Herramienta elegida: \href{https://github.com/}{GitHub}.
\end{itemize}

GitHub es la plataforma web de hospedaje de repositorios por excelencia.
Ofrece todas las funcionalidades de Git, revisión de código,
documentación, \emph{bug tracking}, gestión de tareas, \emph{wikis}, red
social\ldots{} y numerosas integraciones con otros servicios. Es
gratuita para proyectos \emph{open source}.

Utilizamos GitHub como plataforma principal donde hospedamos el código
del proyecto, la gestión de proyecto (gracias a ZenHub) y la
documentación. Además, el repositorio está integrado con varios
servicios de integración continua.

\section{Gestión del proyecto}\label{gestion-del-proyecto}

\begin{itemize}
\tightlist
\item
  Herramientas consideradas: \href{https://www.zenhub.com/}{ZenHub},
  \href{https://trello.com/}{Trello}, \href{https://waffle.io/}{Waffle},
  \href{https://www.versionone.com/}{VersionOne},
  \href{https://xp-dev.com/}{XP-Dev} y \href{https://github.com/}{GitHub
  Projects}.
\item
  Herramienta elegida: \href{https://www.zenhub.com/}{ZenHub}.
\end{itemize}

ZenHub es una herramienta de gestión de proyectos totalmente integrada
en GitHub. Proporciona un tablero canvas en donde cada tarea
representada se corresponde con un \emph{issue} nativo de GitHub. Cada
tarea se puede priorizar dependiendo de su posición en la lista, se le
puede asignar una estimación, uno o varios responsables y el
\emph{sprint} al que pertenece. ZenHub también permite visualizar el
gráfico \emph{burndown} de cada \emph{sprint}. Es gratuita para
proyectos pequeños (máx. 5 colaboradores) o proyectos \emph{open
source}.

\section{Comunicación}\label{comunicacion}

\begin{itemize}
\tightlist
\item
  Herramientas consideradas: email y
  \href{https://gobees.slack.com/}{Slack}.
\item
  Herramienta elegida: \href{https://gobees.slack.com/}{Slack}.
\end{itemize}

Slack es una herramienta de colaboración de equipos que ofrece salas de
chat, mensajes directos y llamadas VoIP. Posee un buscador que permite
encontrar todo el contenido generado dentro de Slack. Además, ofrece un
gran número de integraciones con otros servicios. En nuestro proyecto
vamos a utilizar la integración con GitHub para crear un canal que sirva
de \emph{log} de todas las acciones realizadas en GitHub. Slack ofrece una
versión gratuita que provee las características principales.

\section{Entorno de desarrollo integrado
(IDE)}\label{entorno-de-desarrollo-integrado-ide}

\subsection{Java}\label{java}

\begin{itemize}
\tightlist
\item
  Herramientas consideradas:
  \href{https://www.jetbrains.com/idea/}{IntelliJ IDEA} y
  \href{https://eclipse.org/}{Eclipse}.
\item
  Herramienta elegida: \href{https://www.jetbrains.com/idea/}{IntelliJ
  IDEA}.
\end{itemize}

IntelliJ IDEA es un IDE para Java desarrollado por JetBrains. Posee un
gran número de herramientas para facilitar el proceso de escritura,
revisión y refactorización del código. Además, permite la integración de
diferentes herramientas y posee un ecosistema de \emph{plugins} para
ampliar su funcionalidad. Su versión \emph{community} está disponible
bajo la licencia Apache 2. Aunque también es posible adquirir la versión
\emph{Ultimate} gratuitamente si se es estudiante.

\subsection{Android}\label{android}

\begin{itemize}
\tightlist
\item
  Herramientas consideradas:
  \href{https://developer.android.com/studio/index.html}{Android Studio}
  y \href{https://eclipse.org/}{Eclipse}.
\item
  Herramienta elegida:
  \href{https://developer.android.com/studio/index.html}{Android
  Studio}.
\end{itemize}

Android Studio es el IDE oficial para el desarrollo de aplicaciones
Android. Está basado en IntelliJ IDEA de JetBrains. Proporciona soporte
para Gradle, emulador, editor de \emph{layouts}, refactorizaciones
específicas de Android, herramientas Lint para detectar problemas de
rendimiento, uso, compatibilidad de versión, etc. Se distribuye bajo la
licencia Apache 2.

\subsection{Markdown}\label{markdown}

\begin{itemize}
\tightlist
\item
  Herramientas consideradas: \href{https://stackedit.io/}{StackEdit} y
  \href{http://pad.haroopress.com/}{Haroopad}.
\item
  Herramienta elegida: \href{http://pad.haroopress.com/}{Haroopad}.
\end{itemize}

Haroopad es un editor de documentos Markdown. Soporta Github Flavored
Markdown y Mathematics Expression, además de contar con un gran número
de extensiones. Se distribuye bajo licencia GNU GPL v3.0.

\subsection{LaTeX}\label{latex}

\begin{itemize}
\tightlist
\item
  Herramientas consideradas:
  \href{https://www.sharelatex.com/}{ShareLaTeX} y
  \href{http://www.xm1math.net/texmaker/}{Texmaker}.
\item
  Herramienta elegida:
  \href{http://www.xm1math.net/texmaker/}{Texmaker}.
\end{itemize}

Texmaker es un editor gratuito y multiplataforma para \LaTeX. Integra la
mayoría de herramientas necesarias para la escritura de documentos en
\LaTeX (PdfLaTeX , BibTeX, makeindex, etx). Además, incluye corrector
ortográfico, auto-completado, resaltado de sintaxis, visor de PDFs
integrado, etc. Está licenciado bajo GNU GPL v2.

\section{Documentación}\label{documentacion}

\begin{itemize}
\tightlist
\item
  Herramientas consideradas:
  \href{https://www.latex-project.org/}{LaTeX},
  \href{http://daringfireball.net/projects/markdown/}{Markdown},
  \href{http://docutils.sourceforge.net/docs/ref/rst/restructuredtext.html}{reStructuredText} y  
  \href{https://products.office.com/es-es/word}{Microsoft Word}
\item
  Herramienta elegida:
  \href{http://daringfireball.net/projects/markdown/}{Markdown} +
  \href{http://docutils.sourceforge.net/docs/ref/rst/restructuredtext.html}{reStructuredText} +
  \href{https://www.latex-project.org/}{\LaTeX}.
\end{itemize}

La documentación se ha desarrollado en Markdown y reStructuredText para integrarla con el
servicio de documentación continua \href{https://readthedocs.org/}{Read
the Docs}. Una vez terminada, se ha exportado a \LaTeX utilizando el
conversor \href{http://pandoc.org/}{Pandoc}.

Markdown es un lenguaje de marcado ligero en texto plano que puede ser
exportado a numerosos formatos como HTML o PDF. Su filosofía es que el
lenguaje de marcado sea fácil de escribir y leer. Markdown es
ampliamente utilizado para la escritura de archivos README, en foros
como StackOverflow o en herramientas de comunicación como Slack.

reStructuredText es uno de los leguajes de marcado ligero en los que se 
inspiró Markdown. Su principal aplicación es la escritura de documentación 
de Python junto con el sistema de generación de documentación Sphinx.

\LaTeX es un sistema de composición de textos que genera documentos con
una alta calidad tipográfica. Es ampliamente utilizado para la
generación de artículos y libros científicos, principalmente por su
potencia a la hora de representar expresiones matemáticas.

\section{Servicios de integración
continua}\label{servicios-de-integraciuxf3n-continua}

\subsection{Compilación y testeo}\label{compilacion-y-testeo}

\begin{itemize}
\tightlist
\item
  Herramientas consideradas: \href{https://travis-ci.org/}{TravisCI} y
  \href{https://circleci.com/}{CircleCI}.
\item
  Herramienta elegida: \href{https://travis-ci.org/}{TravisCI}.
\end{itemize}

Travis es una plataforma de integración continua en la nube para
proyectos alojados en GitHub. Permite realizar una \emph{build} del
proyecto y testearla automáticamente cada vez que se realiza un
\emph{commit}, devolviendo un informe con los resultados. Es gratuita
para proyectos \emph{open source}.

\subsection{Cobertura de código}\label{cobertura-de-codigo}

\begin{itemize}
\tightlist
\item
  Herramientas consideradas: \href{https://coveralls.io/}{Coveralls} y
  \href{https://codecov.io/}{Codecov}.
\item
  Herramienta elegida: \href{https://codecov.io/}{Codecov}.
\end{itemize}

Codecov es una herramienta que permite medir el porcentaje de código que
está cubierto por un test. Además, realiza representaciones visuales de
la cobertura y gráficos de su evolución. Posee una extensión de
navegador para GitHub que permite visualizar por cada archivo de código
que líneas están cubiertas por un test y cuáles no. Es gratuita para
proyectos \emph{open source}.

\subsection{Calidad del código}\label{calidad-del-codigo}

\begin{itemize}
\tightlist
\item
  Herramientas consideradas:
  \href{https://codeclimate.com/}{Codeclimate},
  \href{https://sonarqube.com/}{SonarQube} y
  \href{https://www.codacy.com/}{Codacy}.
\item
  Herramientas elegidas: \href{https://codeclimate.com/}{Codeclimate} y
  \href{https://sonarqube.com/}{SonarQube}.
\end{itemize}

Codeclimate es una herramienta que realiza revisiones de código
automáticamente. Es gratuita para proyectos \emph{open source}. En
nuestro proyecto hemos activado los siguientes motores de chequeo:
\href{https://docs.codeclimate.com/docs/checkstyle}{checkstyle},
\href{https://docs.codeclimate.com/docs/fixme}{fixme},
\href{https://docs.codeclimate.com/docs/markdownlint}{markdownlint} y
\href{https://docs.codeclimate.com/docs/pmd}{pmd}.

SonarQube es una plataforma de código abierto para la revisión continua
de la calidad de código. Permite detectar código duplicado, violaciones
de estándares, cobertura de tests unitarios, \emph{bugs} potenciales,
etc.

\subsection{Revisión de dependencias}\label{revision-de-dependencias}

\href{https://www.versioneye.com/}{VersionEye} es una herramienta que monitoriza las dependencias del
proyecto y envía notificaciones cuando alguna de estas está
desactualizada, es vulnerable o viola la licencia del proyecto. Posee
una versión gratuita con ciertas limitaciones.

\subsection{Documentación continua}\label{documentacion-continua}

\href{https://readthedocs.org/}{Read the Docs} es un servicio de documentación continua que permite crear
y hospedar una página web generada a partir de los distintos ficheros
Markdown o reStructuredText de la documentación. Cada vez que se realiza
un \emph{commit} en el repositorio se actualiza la versión hospedada. La
página web posee un buscador, da soporte para diferentes versiones del
proyecto y soporta internacionalización. Además, permite exportar la
documentación en varios formatos (pdf, epub, html, etc.). El servicio es
totalmente gratuito, sostenido por donaciones y subscripciones
\emph{Gold}.

\section{Sistemas de construcción automática del
\emph{software}}\label{sistemas-de-construccion-automuxe1tica-del-software}

\subsection{Maven}\label{maven}

\href{https://maven.apache.org/}{Maven} es una herramienta para
automatizar el proceso de construcción del \emph{software} (compilación,
testeo, empaquetado, etc.) enfocada a proyectos Java. Básicamente
describe cómo se tiene que construir el \emph{software} y cuáles son sus
dependencias.

\subsection{Gradle}\label{gradle}

\href{https://gradle.org/}{Gradle} es una herramienta similar a Maven
pero basada en el lenguaje de programación orientado a objetos Groovy.
El sistema de construcción de Android Studio está basado en Gradle y es
actualmente el único soportado de forma oficial para Android.

\section{Librerías}\label{libreruxedas}

\subsection{\texorpdfstring{\emph{Android Support
Library}}{Android Support Library}}\label{android-support-library}

La
\href{https://developer.android.com/topic/libraries/support-library/}{librería
de soporte de Android} facilita algunas características que no se
incluyen en el \emph{framework} oficial. Proporciona compatibilidad a
versiones antiguas con las últimas características, incluye elementos
para la interfaz adicionales y utilidades extra.

\subsection{Espresso}\label{espresso}

\href{https://google.github.io/android-testing-support-library/docs/espresso/}{Espresso}
es un framework de \emph{testing} para Android incluido en la librería
de soporte para \emph{testing} en Android. Provee una API para escribir
UI test que simulen las interacciones de usuario con la app.

\subsection{Google Guava}\label{google-guava}

\href{https://github.com/google/guava}{Google Guava} agrupa un conjunto
de librerías comunes para Java. Proporciona utilidades básicas para
tareas cotidianas, una extensión del \emph{Java collections framework}
(JCF) y otras extensiones como programación funcional, almacenamiento en
caché, objetos de rango o \emph{hashing}.

\subsection{Google Play Services}\label{google-play-services}

\href{https://developers.google.com/android/guides/overview}{Google Play
Services} es una librería que permite a las aplicaciones de terceros
utilizar características de aplicaciones de Google como Maps, Google+,
etc. En nuestro caso se ha hecho uso de su servicio de localización, que
utiliza varias fuentes de datos (GPS, red y \emph{wifi}) para ubicar el
dispositivo rápidamente.

\subsection{JavaFX}\label{javafx}

\href{http://docs.oracle.com/javase/8/javase-clienttechnologies.htm}{JavaFX}
es una librería para la creación de interfaces gráficas en Java.

\subsection{JUnit}\label{junit}

\href{http://junit.org/junit4/}{JUnit} es un \emph{framework} para Java
utilizado para realizar pruebas unitarias.

\subsection{Material Design}\label{material-design}

\href{https://material.io/guidelines/}{Material Design} es una guía de
estilos enfocada a la plataforma Android, pero aplicable a cualquier
otra plataforma. Fue presentada en el Google I/O 2014 y se adoptó en
Android a partir de la versión 5.0 (Lollipop). Se basa en objetos
materiales, piezas colocadas en un espacio (lugar) y con un tiempo
(movimiento) determinado.

\subsection{Mockito}\label{mockito}

\href{http://mockito.org/}{Mockito} es un \emph{framework} de
\emph{mocking} que permite crear objetos \emph{mock} fácilmente. Estos
objetos simulan parte del comportamiento de una clase. Mockito está
basado en EasyMock, mejorando su sintaxis haciendo los test más simples
y fáciles de leer y con mensajes de error descriptivos.

\subsection{MPAndroidChart}\label{mpandroidchart}

\href{https://github.com/PhilJay/MPAndroidChart}{MPAndroidChart} es una
librería para la creación de gráficos en Android.

\subsection{OpenCV}\label{opencv}

\href{www.opencv.org}{OpenCV} es un paquete \emph{Open Source} de visión
artificial que contiene más de 2500 librerías de procesamiento de
imágenes y visión artificial, escritas en C/C++ a bajo/medio nivel. Se
distribuye gratuitamente bajo una licencia \emph{BSD} desde hace más de
una década. Posee una comunidad de más de 50.000 usuarios alrededor de
todo el mundo y se ha descargado más de 8 millones de veces.

Aunque OpenCV está escrito en C/C++ posee \emph{wrappers} para varias
plataformas, entre ellas Android, en donde da soporte a las principales
arquitecturas de CPU. Desde hace unos años, también soporta CUDA para el
desarrollo en GPU tanto en escritorio como en móvil, aunque en esta
última el soporte es todavía reducido.

\subsection{OpenWeatherMaps}\label{openweathermaps}

\href{http://openweathermap.org/}{OpenWeatherMap} es un servicio online
que proporciona información meteorológica. Está inspirado en
OpenStreetMap y su filosofía de hacer accesible la información a la
gente de forma gratuita. Utiliza distintas fuentes de datos desde
estaciones meteorológicas oficiales, de aeropuertos, radares e incentiva
a los propietarios de estaciones meteorológicas a conectarlas a su red.
Proporciona una API que permite realizar hasta 60 llamadas por segundo
de forma gratuita.

\subsection{PowerMock}\label{powermock}

\href{https://github.com/powermock/powermock}{PowerMock} es una librería
de \emph{testing} que permite la creación de \emph{mocks} de métodos
estáticos, constructores, clases finales o métodos privados.

\subsection{Realm}\label{realm}

\href{https://realm.io/products/realm-mobile-database/}{Realm} es una
base de datos orientada a objetos enfocada a dispositivos móviles. Se
definen como la alternativa a SQLite y presumen de ser más rápidos que
cualquier ORM e incluso que SQLite puro. Posee una API muy intuitiva que
facilita en gran medida el acceso a datos.

\newpage
\section{Página web}\label{pagina-web}

\subsection{GitHub Pages}\label{github-pages}

\href{https://pages.github.com/}{GitHub Pages} es un servicio de hosting
estático que permite a proyectos que utilicen un repositorio de GitHub 
hospedar su página web en el propio repositorio. Permite utilizar Jekyll, un generador de sitios
estáticos. No soporta tecnologías del lado de servidor como PHP, Ruby,
Python, etc.

\subsection{Bootstrap}\label{bootstrap}

\href{http://getbootstrap.com/}{Bootstrap} es un \emph{framework} para
desarrollo \emph{front-end}. Contiene una serie de componentes ya
implementados que facilitan y agilizan el diseño. Está desarrollado
siguiendo la filosofía \emph{mobile first}.

\section{Otras herramientas}\label{otras-herramientas}

\subsection{Mendeley}\label{mendeley}

\href{https://www.mendeley.com/}{Mendeley} es un gestor de referencias
bibliográficas. Permite añadir referencias de varias formas, visualizar
los documentos, etiquetarlos, compartirlos, etc. Posteriormente se puede
exportar todo el catálogo a un fichero BibTex para ser utilizadas desde
\LaTeX.

\subsection{Creately}\label{creately}

\href{https://creately.com/}{Creately} es una aplicación web que permite
crear todo tipo de diagramas altamente personalizables. Aunque posee una
versión gratuita limitada, se optó por pagar un mes de subscripción al
valorar que realmente iba a ser de utilidad.

\capitulo{5}{Aspectos relevantes del desarrollo del proyecto}

Este apartado pretende recoger los aspectos más interesantes del desarrollo del proyecto, comentados por los autores del mismo.
Debe incluir desde la exposición del ciclo de vida utilizado, hasta los detalles de mayor relevancia de las fases de análisis, diseño e implementación.
Se busca que no sea una mera operación de copiar y pegar diagramas y extractos del código fuente, sino que realmente se justifiquen los caminos de solución que se han tomado, especialmente aquellos que no sean triviales.
Puede ser el lugar más adecuado para documentar los aspectos más interesantes del diseño y de la implementación, con un mayor hincapié en aspectos tales como el tipo de arquitectura elegido, los índices de las tablas de la base de datos, normalización y desnormalización, distribución en ficheros3, reglas de negocio dentro de las bases de datos (EDVHV GH GDWRV DFWLYDV), aspectos de desarrollo relacionados con el WWW...
Este apartado, debe convertirse en el resumen de la experiencia práctica del proyecto, y por sí mismo justifica que la memoria se convierta en un documento útil, fuente de referencia para los autores, los tutores y futuros alumnos.

\capitulo{6}{Trabajos relacionados}

Como se comentó en la introducción, los intentos de automatizar el
proceso de monitorización de la actividad de una colmena se remontan
hasta principios del siglo pasado. Sin embargo, no es hasta 2008 cuando
se introduce la visión artificial en este campo. A continuación, se
exponen los artículos científicos relacionados publicados hasta la
fecha, así como proyectos con objetivos similares.

\section{Artículos científicos}\label{artuxedculos-cientuxedficos}

\subsection{\emph{Video Monitoring of Honey Bee Colonies at the Hive
Entrance}}\label{video-monitoring-of-honey-bee-colonies-at-the-hive-entrance}

Se trata del primer artículo publicado sobre el tema (año 2008). Los
autores fueron Jason Campbell, Lily Mummert y Rahul Sukthankar del
\emph{Intel Research Pittsburgh}. En él proponen un método de visión
artificial para monitorizar las entradas y salidas de abejas en una
colmena, consiguiendo diferenciar las que entran de las que salen. Se
describen los desafíos técnicos que supuso y la solución a la que
llegaron finalmente \citep{art:campbell2008}.

\subsection{\emph{Detecting and tracking honeybees in 3D at the beehive
entrance using stereo
vision}}\label{detecting-and-tracking-honeybees-in-3d-at-the-beehive-entrance-using-stereo-vision}

En 2013, Guillaume Chiron, Petra Gomez-Krämer y Ménard Michel publicaron
un artículo en \emph{EURASIP Journal on Image and Video Processing,}
donde proponían un método para la monitorización de abejas a la entrada
de una colmena basado en un sistema de tiempo real con visión
estereoscópica. Gracias al cual podían obtener una representación en
tres dimensiones de las trayectorias de las abejas
\citep{art:chiron2013}.

\subsection{\emph{Image Processing for Honey Bee hive Health
Monitoring}}\label{image-processing-for-honey-bee-hive-health-monitoring}

El último artículo publicado data del año 2015 por Rahman Tashakkori y
Ahmad Ghadiri de la \emph{Appalachian State University}. En él, mejoran
el método de detección propuesto en \citep{art:campbell2008} y lo
utilizan para estimar el número de abejas que habrá en un instante de
tiempo dado \citep{art:tashakkori2015}.

\subsection{Comparativa sobre las técnicas utilizadas}\label{comparacion-articulos}

A continuación, se muestran las diferentes técnicas de detección de movimiento,
conteo de abejas y \emph{tracking} utilizadas en los tres artículos 
anteriores y se comparan con las utilizadas en el proyecto.

\tablaSmall{Comparativa métodos de detección de movimiento.}{l l l l l}{comparativa-1}
{ Artículo & Año & Citas & Detección de movimiento \\}{ 
{\citep{art:campbell2008}}   & 2008 & 24 & \emph{Adaptative background subtraction}. \\
{\citep{art:chiron2013}}     & 2013 & 8  & \specialcell{\emph{Adaptative background subtraction}\\\emph{with depth information}.} \\
{\citep{art:tashakkori2015}} & 2015 & 0  & \specialcell{\emph{Averaging a background with}\\\emph{illumination invariant method}.}\\
GoBees                       & 2017 & 0  & \specialcell{\emph{Mixture of Gaussians method}\\(\texttt{BackgroundSubtractorMOG2}).}     \\
} 

\tablaSmall{Comparativa métodos de conteo de abejas y \emph{tracking}.}{l l l}{comparativa-2}
{ Artículo & Conteo de abejas & \emph{Tracking} \\}{ 
{\citep{art:campbell2008}} & \emph{Template-based method.} & \specialcell{\emph{Maximum weighted bipartite}\\\emph{graph matching.}} \\
{\citep{art:chiron2013}} & \specialcell{\emph{Hybrid 3D intensity}\\\emph{depth method.}} & \specialcell{\emph{Kalman filter }y\\\emph{Global Nearest Neighbor}.} \\
{\citep{art:tashakkori2015}} & \emph{Area-based method.} & No \\
GoBees & \emph{Area-based method.} & No \\
} 
\newpage

\section{Proyectos}\label{proyectos}

\subsection{EyesOnHives}\label{eyesonhives}

EyesOnHives es el principal competidor del proyecto. Se trata de un
producto comercial cuyo fin es la monitorización del estado de salud de
las colmenas mediante su actividad de vuelo. Integra un \emph{hardware}
específico que se encarga de la captación de imágenes y una plataforma
en la nube que las procesa y permite el acceso a los datos.

\begin{itemize}
\tightlist
\item
  Web del proyecto: \url{http://www.keltronixinc.com}
\end{itemize}

\subsection{HiveTool}\label{hivetool}

Se trata de un proyecto \emph{OpenSource} que ofrece un conjunto de
herramientas para monitorizar distintos parámetros de una colmena. Una
de estas herramientas es ``Bee Counter'', un contador de abejas por
visión artificial desarrollado sobre una Raspberry Pi.

\begin{itemize}
\tightlist
\item
  Web del proyecto: \url{http://hivetool.org}
\end{itemize}

\section{Fortalezas y debilidades del
proyecto}\label{fortalezas-y-debilidades-del-proyecto}

\begin{table}[H]
\centering
\begin{tabular}{lccc}
\toprule
Características                 & GoBees     & EyesOnHives & HiveTool   \\
\midrule
No requiere \emph{hardware} específico & \cellcolor{green!25} {$\checkmark$} & \cellcolor{red!25} {$\times$} & \cellcolor{red!25} {$\times$} \\
Instalación sencilla            & \cellcolor{green!25} {$\checkmark$} & \cellcolor{green!25} {$\checkmark$}  & \cellcolor{red!25} {$\times$} \\
Procesamiento en local          & \cellcolor{green!25} {$\checkmark$} & \cellcolor{yellow!25} Parcial & \cellcolor{green!25} {$\checkmark$}  \\
No requiere \emph{wifi}         & \cellcolor{green!25} {$\checkmark$} & \cellcolor{red!25} {$\times$} & \cellcolor{green!25} {$\checkmark$}  \\
No requiere red eléctrica       & \cellcolor{green!25} {$\checkmark$} & \cellcolor{red!25} {$\times$}  & \cellcolor{green!25} {$\checkmark$}  \\
Localización GPS                & \cellcolor{green!25} {$\checkmark$} & \cellcolor{red!25} {$\times$} & \cellcolor{red!25} {$\times$}        \\
Gratuito                        & \cellcolor{green!25} {$\checkmark$} & \cellcolor{red!25} {$\times$}  & \cellcolor{green!25} {$\checkmark$}  \\
Plataformas                     & Android    & Web App     & Linux     \\
\bottomrule
\end{tabular}
\caption{Comparativa de las características de los proyectos.}
\label{comparativa-proyectos}
\end{table}

Las principales fortalezas del proyecto son:

\begin{itemize}
\tightlist
\item
  No se necesita adquirir ningún \emph{hardware} específico como en el
  resto de proyectos, simplemente se necesita un \emph{smartphone} con
  Android. Esto hace el proyecto mucho más accesible a los potenciales
  usuarios.
\item
  La instalación es muy sencilla. Únicamente se requiere un trípode o
  cualquier otro tipo de soporte que permita sujetar el
  \emph{smartphone} en posición cenital.
\item
  El procesamiento de las imágenes se realiza en local no en un
  servidor. Considerando que los colmenares suelen estar en medio del
  monte, no podemos requerir una conexión \emph{wifi} como necesita
  EyesOnHives y el envío de vídeo mediante tecnologías 3G/4G supondría
  un coste económico muy elevado.
\item
  No requiere estar conectado a la red eléctrica. El \emph{smartphone}
  cuenta con su propia batería. El consumo de la aplicación no es muy
  elevado al estar la pantalla apagada durante la monitorización. Aun
  así, se pueden utilizar \emph{powerbanks} (baterías portátiles) en
  caso de ser necesarios.
\item
  El \emph{smartphone} tiene integradas varias tecnologías de
  transmisión de información. Lo que da la posibilidad de crear una
  plataforma que centralice la recogida de datos de varios dispositivos
  sin importar su localización.
\item
  Relacionado con el punto anterior, el \emph{smartphone} nos permite
  estar conectados a internet, posibilitándonos ampliar la información
  que maneja nuestra aplicación. Por ejemplo, podemos acceder a la
  información meteorológica en tiempo real.
\item
  El GPS del \emph{smartphone} nos permite localizar geográficamente la
  monitorización y, por tanto, la información meteorológica. Además,
  puede ser de utilidad en caso de robo, gracias a aplicaciones como
  \emph{Android Device Manager}, Cerberus, etc. que permiten localizar
  el dispositivo de forma remota.
\end{itemize}

Las principales debilidades son:

\begin{itemize}
\tightlist
\item
  Actualmente solo se encuentra disponible para Android. Aunque en una
  segunda fase del proyecto se creará una plataforma en la nube que
  centralice todos los datos y una aplicación web que permita acceder a
  ellos.
\item
  El utilizar un \emph{smartphone} como soporte \emph{hardware} tiene
  sus ventajas, pero también sus inconvenientes. La cámara no tiene el
  mismo rendimiento que una cámara diseñada específicamente para esta
  tarea. Esto nos ha limitado en las técnicas de visión artificial que
  hemos podido aplicar, por no disponer de imágenes con la suficiente
  nitidez.
\end{itemize}

\capitulo{7}{Conclusiones y Líneas de trabajo futuras}

Todo proyecto debe incluir las conclusiones que se derivan de su desarrollo. Éstas pueden ser de diferente índole, dependiendo de la tipología del proyecto, pero normalmente van a estar presentes un conjunto de conclusiones relacionadas con los resultados del proyecto y un conjunto de conclusiones técnicas. 
Además, resulta muy útil realizar un informe crítico indicando cómo se puede mejorar el proyecto, o cómo se puede continuar trabajando en la línea del proyecto realizado. 



\bibliography{bibliografia}
\bibliographystyle{plainnat}

\end{document}