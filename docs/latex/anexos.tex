\documentclass[a4paper,11pt,oneside]{memoir}

% Castellano
\usepackage[spanish,es-tabla]{babel}
\selectlanguage{spanish}
\usepackage[utf8]{inputenc}
\usepackage{placeins}
\usepackage{float}
\usepackage{eurosym}

\usepackage{longtable,booktabs}
\RequirePackage[table]{xcolor}
\RequirePackage{xtab}
\RequirePackage{multirow}

% Landscape
\usepackage{pdflscape}

% Multi-page tables using
\usepackage{longtable}
\usepackage{tabularx}

% Cell with line break (e.g. \specialcell{Foo\\bar})
\newcommand{\specialcell}[2][c]{%
  \begin{tabular}[#1]{@{}l@{}}#2\end{tabular}}

% Mathematic font
\usepackage{amsfonts}

% Color

% Bibliography management
\usepackage[numbers,sort]{natbib}

% Links
\usepackage[colorlinks]{hyperref}
\hypersetup{
	colorlinks,
	linkcolor={green!40!black},
	citecolor={blue!50!black},
	urlcolor={blue!80!black}
}

% Ecuaciones
\usepackage{amsmath}

% Rutas de fichero / paquete
\newcommand{\ruta}[1]{{\sffamily #1}}

% Párrafos
\nonzeroparskip

% Listas estrechas
\providecommand{\tightlist}{%
  \setlength{\itemsep}{0pt}\setlength{\parskip}{0pt}}

% Imagenes
\usepackage{graphicx}
\newcommand{\imagen}[2]{
	\begin{figure}[!h]
		\centering
		\includegraphics[width=0.9\textwidth]{#1}
		\caption{#2}\label{fig:#1}
	\end{figure}
	\FloatBarrier
}

\newcommand{\imagenAncho}[3]{
	\begin{figure}[H]
		\centering
		\includegraphics[width=#3\textwidth]{#1}
		\caption{#2}\label{fig:#1}
	\end{figure}
	\FloatBarrier
}

\newcommand{\imagenflotante}[2]{
	\begin{figure}%[!h]
		\centering
		\includegraphics[width=0.9\textwidth]{#1}
		\caption{#2}\label{fig:#1}
	\end{figure}
}



% El comando \figura nos permite insertar figuras comodamente, y utilizando
% siempre el mismo formato. Los parametros son:
% 1 -> Porcentaje del ancho de página que ocupará la figura (de 0 a 1)
% 2 --> Fichero de la imagen
% 3 --> Texto a pie de imagen
% 4 --> Etiqueta (label) para referencias
% 5 --> Opciones que queramos pasarle al \includegraphics
% 6 --> Opciones de posicionamiento a pasarle a \begin{figure}
\newcommand{\figuraConPosicion}[6]{%
  \setlength{\anchoFloat}{#1\textwidth}%
  \addtolength{\anchoFloat}{-4\fboxsep}%
  \setlength{\anchoFigura}{\anchoFloat}%
  \begin{figure}[#6]
    \begin{center}%
      \Ovalbox{%
        \begin{minipage}{\anchoFloat}%
          \begin{center}%
            \includegraphics[width=\anchoFigura,#5]{#2}%
            \caption{#3}%
            \label{#4}%
          \end{center}%
        \end{minipage}
      }%
    \end{center}%
  \end{figure}%
}

%
% Comando para incluir imágenes en formato apaisado (sin marco).
\newcommand{\figuraApaisadaSinMarco}[5]{%
  \begin{figure}%
    \begin{center}%
    \includegraphics[angle=90,height=#1\textheight,#5]{#2}%
    \caption{#3}%
    \label{#4}%
    \end{center}%
  \end{figure}%
}
% Para las tablas
\newcommand{\otoprule}{\midrule [\heavyrulewidth]}
%
% Nuevo comando para tablas pequeñas (menos de una página).
\newcommand{\tablaSmall}[5]{%
 \begin{table}[H]
  \begin{center}
   \rowcolors {2}{gray!35}{}
   \begin{tabular}{#2}
    \toprule
    #4
    \otoprule
    #5
    \bottomrule
   \end{tabular}
   \caption{#1}
   \label{tabla:#3}
  \end{center}
 \end{table}
}

%
% Nuevo comando para tablas pequeñas (menos de una página).
\newcommand{\tablaSmallSinColores}[5]{%
 \begin{table}[H]
  \begin{center}
   \begin{tabular}{#2}
    \toprule
    #4
    \otoprule
    #5
    \bottomrule
   \end{tabular}
   \caption{#1}
   \label{tabla:#3}
  \end{center}
 \end{table}
}

\newcommand{\tablaApaisadaSmall}[5]{%
\begin{landscape}
  \begin{table}
   \begin{center}
    \rowcolors {2}{gray!35}{}
    \begin{tabular}{#2}
     \toprule
     #4
     \otoprule
     #5
     \bottomrule
    \end{tabular}
    \caption{#1}
    \label{tabla:#3}
   \end{center}
  \end{table}
\end{landscape}
}

%
% Nuevo comando para tablas grandes con cabecera y filas alternas coloreadas en gris.
\newcommand{\tabla}[6]{%
  \begin{center}
    \tablefirsthead{
      \toprule
      #5
      \otoprule
    }
    \tablehead{
      \multicolumn{#3}{l}{\small\sl continúa desde la página anterior}\\
      \toprule
      #5
      \otoprule
    }
    \tabletail{
      \hline
      \multicolumn{#3}{r}{\small\sl continúa en la página siguiente}\\
    }
    \tablelasttail{
      \hline
    }
    \bottomcaption{#1}
    \rowcolors {2}{gray!35}{}
    \begin{xtabular}{#2}
      #6
      \bottomrule
    \end{xtabular}
    \label{tabla:#4}
  \end{center}
}

%
% Nuevo comando para tablas grandes con cabecera.
\newcommand{\tablaSinColores}[6]{%
  \begin{center}
    \tablefirsthead{
      \toprule
      #5
      \otoprule
    }
    \tablehead{
      \multicolumn{#3}{l}{\small\sl continúa desde la página anterior}\\
      \toprule
      #5
      \otoprule
    }
    \tabletail{
      \hline
      \multicolumn{#3}{r}{\small\sl continúa en la página siguiente}\\
    }
    \tablelasttail{
      \hline
    }
    \bottomcaption{#1}
    \begin{xtabular}{#2}
      #6
      \bottomrule
    \end{xtabular}
    \label{tabla:#4}
  \end{center}
}

%
% Nuevo comando para tablas grandes sin cabecera.
\newcommand{\tablaSinCabecera}[5]{%
  \begin{center}
    \tablefirsthead{
      \toprule
    }
    \tablehead{
      \multicolumn{#3}{l}{\small\sl continúa desde la página anterior}\\
      \hline
    }
    \tabletail{
      \hline
      \multicolumn{#3}{r}{\small\sl continúa en la página siguiente}\\
    }
    \tablelasttail{
      \hline
    }
    \bottomcaption{#1}
  \begin{xtabular}{#2}
    #5
   \bottomrule
  \end{xtabular}
  \label{tabla:#4}
  \end{center}
}



\definecolor{cgoLight}{HTML}{EEEEEE}
\definecolor{cgoExtralight}{HTML}{FFFFFF}

%
% Nuevo comando para tablas grandes sin cabecera.
\newcommand{\tablaSinCabeceraConBandas}[5]{%
  \begin{center}
    \tablefirsthead{
      \toprule
    }
    \tablehead{
      \multicolumn{#3}{l}{\small\sl continúa desde la página anterior}\\
      \hline
    }
    \tabletail{
      \hline
      \multicolumn{#3}{r}{\small\sl continúa en la página siguiente}\\
    }
    \tablelasttail{
      \hline
    }
    \bottomcaption{#1}
    \rowcolors[]{1}{cgoExtralight}{cgoLight}

  \begin{xtabular}{#2}
    #5
   \bottomrule
  \end{xtabular}
  \label{tabla:#4}
  \end{center}
}


















\graphicspath{ {../img/} }

% Capítulos
\chapterstyle{bianchi}
\newcommand{\capitulo}[2]{
	\setcounter{chapter}{#1}
	\setcounter{section}{0}
	\chapter*{#2}
	\addcontentsline{toc}{chapter}{#2}
	\markboth{#2}{#2}
}

% Apéndices
\renewcommand{\appendixname}{Apéndice}
\renewcommand*\cftappendixname{\appendixname}

\newcommand{\apendice}[1]{
	%\renewcommand{\thechapter}{A}
	\chapter{#1}
}

\renewcommand*\cftappendixname{\appendixname\ }

% Formato de portada
\makeatletter
\usepackage{xcolor}
\newcommand{\tutor}[1]{\def\@tutor{#1}}
\newcommand{\course}[1]{\def\@course{#1}}
\definecolor{cpardoBox}{HTML}{E6E6FF}
\def\maketitle{
  \null
  \thispagestyle{empty}
  % Cabecera ----------------
\noindent\includegraphics[width=\textwidth]{cabecera}\vspace{1cm}%
  \vfill
  % Título proyecto y escudo informática ----------------
  \colorbox{cpardoBox}{%
    \begin{minipage}{.8\textwidth}
      \vspace{.5cm}\Large
      \begin{center}
      \textbf{TFG del Grado en Ingeniería Informática}\vspace{.6cm}\\
      \textbf{\LARGE\@title{}}
      \end{center}
      \vspace{.2cm}
    \end{minipage}

  }%
  \hfill\begin{minipage}{.20\textwidth}
    \includegraphics[width=\textwidth]{escudoInfor}
  \end{minipage}
  \vfill
  % Datos de alumno, curso y tutores ------------------
  \begin{center}%
  {%
    \noindent\LARGE
    Presentado por \@author{}\\ 
    en la Universidad de Burgos --- \@date{}\\
    Tutores: \@tutor{}\\
  }%
  \end{center}%
  \null
  \cleardoublepage
  }
\makeatother


% Datos de portada
\title{{\Huge GoBees}\\[0.5cm]Monitorización del estado de una colmena mediante la cámara de un smartphone.}
\author{David Miguel Lozano}
\tutor{Dr. José Francisco Díez Pastor\\y Dr. Raúl Marticorena Sánchez}
\date{\today}

\begin{document}

\maketitle



\cleardoublepage



%%%%%%%%%%%%%%%%%%%%%%%%%%%%%%%%%%%%%%%%%%%%%%%%%%%%%%%%%%%%%%%%%%%%%%%%%%%%%%%%%%%%%%%%



\frontmatter


\clearpage

% Indices
\tableofcontents

\clearpage

\listoffigures

\clearpage

%\listoftables

%\clearpage

\mainmatter

\appendix

\apendice{Manuales}

\section{Introducción}

\section{Planificación temporal}

\section{Estudio de viabilidad}

\subsection{Viabilidad económica}

\subsection{Viabilidad legal}



\apendice{Especificación de Requisitos}

\section{Introducción}

\section{Objetivos generales}

\section{Catalogo de requisitos}

\section{Especificación de requisitos}



\apendice{Especificación de diseño}

\section{Introducción}\label{introduccion}

En este anexo se define cómo se han resuelto los objetivos y
especificaciones expuestos con anterioridad. Define los datos que va a
manejar la aplicación, su arquitectura, el diseño de sus interfaces, sus
detalles procedimentales, etc.

\section{Diseño de datos}\label{diseno-de-datos}

La aplicación cuenta con las siguientes entidades:

\begin{itemize}
\tightlist
\item
  \textbf{Colmenar (Apiary)}: tiene un nombre, una imagen, una
  localización y unas notas. A su vez, guarda un registro del tiempo
  meteorológico actual y varios registros del tiempo que hacía cuando se
  realizaron las grabaciones de sus colmenas.
\item
  \textbf{Colmena (Hive)}: tiene un nombre, una imagen y unas notas. A
  su vez, posee varias grabaciones de distintas monitorizaciones de la
  colmena.
\item
  \textbf{Registro (Record)}: se corresponde a la salida del algoritmo
  de conteo al analizar un fotograma. Tiene un \emph{timestamp} y el
  número de abejas que había en el fotograma.
\item
  \textbf{Registro meteorológico (MeteoRecord)}: guarda información
  sobre el estado meteorológico en una localización y un momento dado.
  Tiene un \emph{timestamp}, la localidad, el código correspondiente a
  la condición meteorológica, el icono correspondiente, temperatura,
  presión, humedad, velocidad y dirección del viento, porcentaje de
  nubes, precipitaciones, y nieve.
\end{itemize}
\newpage
\subsection{Diagrama E/R}\label{diagrama-er}

\imagenAncho{er-diagram}{Diagrama E/R.}{1}

\subsection{Diagrama Relacional}\label{diagrama-relacional}

\imagenAncho{relational-diagram}{Diagrama relacional.}{0.8}

\section{Diseño arquitectónico}\label{diseno-arquitectonico}

El hecho de que el proyecto se haya realizado para la plataforma Android
ha condicionado muchas de las decisiones de diseño. Aun así, se han
aplicado una serie de patrones para intentar desacoplar el código lo
máximo posible y así mejorar su testabilidad y mantenibilidad.

\subsection{Model-View-Presenter (MVP)}\label{model-view-presenter-mvp}

Uno de los patrones arquitectónicos que más relevancia está ganando para
el desarrollo de aplicaciones es MVP (\emph{Model-View-Presenter)}. Se
trata de un patrón derivado del MVC (\emph{Model-View-Controler}) cuyo
objetivo es separar la vista del modelo de datos subyacente. MVP
introduce la figura del \emph{presenter} que actúa de mediador entre
estas dos capas. Su segundo objetivo es maximizar la cantidad de código
que se puede testear de forma automática.

MVP divide la aplicación en las siguientes capas:\citep{pattern:mvp}

\begin{itemize}
\tightlist
\item
  \emph{Model}: se corresponde únicamente con el acceso a datos. Se
  encarga de almacenar y proporcionar los diferentes datos que maneja la
  aplicación. En nuestra aplicación se corresponde con el Repositorio.
\item
  \emph{View}: se encarga de la visualización de los datos (del modelo).
  Propaga todas las acciones de usuario al \emph{presenter}. En nuestra
  aplicación se corresponde con los \emph{Fragments}.
\item
  \emph{Presenter}: enlaza las dos capas anteriores. Sincroniza los
  datos mostrados en la vista con los almacenados en el modelo y actúa
  ante los eventos de usuario propagados por la vista. En nuestra
  aplicación se corresponde con los \emph{Presenters}.
\end{itemize}

\imagen{mvp}{Patrón MVP.}

Existen varias variantes sobre cómo implementar MVP en Android. En
nuestro caso, se ha seguido la expuesta Google en Android Architecture
Blueprints \citep{pattern:android_architecture}. En ella se realizan
las siguientes consideraciones:

\begin{itemize}
\tightlist
\item
  Se utilizan las \emph{Activity} como controladores globales que se
  encargan de crear y conectar las vistas con los \emph{presenters}.
\item
  Se utilizan los \emph{Fragment} como vistas ya que proporcionan
  numerosas ventajas cuando se trabaja con múltiples vistas.
\end{itemize}

\subsection{Patrón repositorio}\label{patron-repositorio}

Para la capa del modelo, se ha utilizado el patrón repositorio que
proporciona una abstracción de la implementación del acceso a datos con
el objetivo de que este sea transparente a la lógica de negocio
\citep{pattern:repository}.

En nuestra aplicación existen dos fuentes de datos: por una parte, está
la base de datos local implementada con Realm, y por otra, tenemos la
API remota que nos da acceso a la información meteorológica. Ambas
fuentes son transparentes para los \emph{presenters}.

El repositorio media entre la capa de acceso a datos y la lógica de
negocio de tal forma que no existe ninguna dependencia entre ellas.
Consiguiendo desacoplar, mantener y testear más fácilmente el código y
permitiendo la reutilización del acceso a datos desde cualquier cliente.

\imagen{repository_pattern}{Patrón repositorio.}

\subsection{Inyección de
dependencias}\label{inyeccion-de-dependencias}

A la hora de testear, es muy frecuente necesitar sustituir la
implementación de una clase por otra ``falsa'' que se comporte de una
manera predeterminada para conseguir probar la funcionalidad de manera
aislada. En nuestro caso, para facilitar la labor de testeo nos vimos
obligados a sustituir la base de datos Realm por una base de datos en
memoria. Esta sustitución se realizó mediante la inyección de
dependencias.

La inyección de dependencias es un patrón mediante el cual se
proporcionan todas las dependencias que una clase necesita para su
funcionamiento, en lugar de ser la propia clase quien las cree. Al
separar las dependencias de la propia clase, se posibilita la opción de
sustituir estas por dobles con un comportamiento definido
\citep{wiki:injection}.

Para la implementación de la inyección de dependencias se han utilizado
los \emph{build flavors} que proporciona Gradle. Se crearon dos
\emph{flavors}:

\begin{itemize}
\tightlist
\item
  \texttt{mock}: inyectaba una base de datos en memoria utilizada para el testeo
  de la aplicación.
\item
  \texttt{prod}: inyectaba la base de datos Realm utilizada para producción.
\end{itemize}

\subsection{Arquitectura general}\label{arquitectura-general}

El resultado de la arquitectura tras aplicar los patrones explicados es
el siguiente:

\imagen{architecture}{Arquitectura de la aplicación.}

Para agilizar la navegación por la aplicación se implementó una capa de
caché en el repositorio.

\subsection{Diseño de paquetes}\label{diseno-de-paquetes}

Para la organización de los diferentes archivos que componen la
aplicación no se utilizó la estrategia convencional de paquete por capa
(\emph{package by layer approach}), sino una estrategia de paquete por
característica (\emph{package per feature approach}).

Siguiendo esta estrategia se agruparon todos los archivos relacionados
cada una de las distintas funcionalidades de la aplicación en un mismo
paquete. De esta manera se mejora notablemente la legibilidad y la
modularización de la aplicación, ya que se puede modificar cada
funcionalidad de forma independiente.

Existen dos paquetes excepcionales que no siguen esta convención:

\begin{itemize}
\tightlist
\item
  Paquete \texttt{data}: agrupa toda la capa de modelo.
\item
  Paquete \texttt{utils}: reúne un conjunto de clases de utilidad
  generales que son utilizadas por varias características.
\end{itemize}

El diagrama de paquetes es el siguiente:

\imagenAncho{packages-diagram}{Diagrama de paquetes simplificado.}{1}

El paquete \texttt{feature X} se correspondería con cada paquete de cada
funcionalidad. Se ha representado de esta manera para simplificar el
diagrama.

A continuación, se muestran por separado los paquetes de todas las
funcionalidades:

\imagenAncho{packages-features-diagram}{Paquetes de las diferentes características.}{1}

\begin{itemize}
\tightlist
\item
  \texttt{about}: contiene la funcionalidad de ``Acerca de'' de la
  aplicación. Donde se muestra el autor, licencia, versión de la
  \emph{app}, sitio web, historial de cambio y todas las dependencias
  junto con sus licencias.
\item
  \texttt{addeditapiary}: permite añadir o editar colmenares.
\item
  \texttt{addedithive}: permite añadir o editar colmenas.
\item
  \texttt{apiaries}: permite listar los colmenares y gestionarlos.
\item
  \texttt{apiary}: permite listar las colmenas de un colmenar,
  gestionarlas y mostrar la información relativa al colmenar.
\item
  \texttt{help}: muestra la ayuda de la aplicación.
\item
  \texttt{hive}: permite listar las grabaciones de una colmena,
  gestionarlas y mostrar la información relativa a la colmena.
\item
  \texttt{monitoring}: agrupa toda la funcionalidad de monitorización de
  la actividad de vuelo de una colmena, desde la configuración hasta la
  ejecución del algoritmo.
\item
  \texttt{recording}: permite visualizar los detalles de una determinada
  grabación.
\item
  \texttt{settings}: permite configurar los distintos parámetros de la
  aplicación.
\item
  \texttt{splash}: muestra una pantalla de inicio mientras la aplicación
  carga en memoria los recursos necesarios.
\end{itemize}

\subsection{Diseño de clases}\label{diseno-de-clases}

Aplicando MVP, cada característica clave de la aplicación posee los
siguientes componentes:

\begin{itemize}
\tightlist
\item
  \texttt{FeatureActivity}: funciona como un controlador global que crea la vista
  y el \emph{presenter} y los enlaza.
\item
  \texttt{FeatureContract}: se trata de una interfaz que establece los siguientes
  contratos:

  \begin{itemize}
  \tightlist
  \item
    \texttt{FeatureContract.View}: define la capa \emph{view} para esta
    característica (las únicas funciones que expone a otras capas).
  \item
    \texttt{FeatureContract.Presenter}: define la interacción entre las capas
    \emph{view} y \emph{presenter}. Describe las acciones que pueden ser
    iniciadas desde la vista.
  \end{itemize}
\item
  \texttt{FeatureFragment}: implementación concreta de la capa \emph{view}.
\item
  \texttt{FeaturePresenter}: implementación concreta de la capa \emph{presenter}.
  Escucha las acciones de usuario y actualiza la vista cuando cambia el
  modelo.
\end{itemize}

\imagenAncho{feature-package}{Paquete tipo de una característica.}{0.5}

El diagrama de clases general que muestra cómo se relacionan todos los
componentes de una determinada característica es el siguiente:

\imagenAncho{general-class-diagram}{Diagrama de clases general.}{1}

El único paquete que se diferencia de la estructura expuesta es el
paquete \texttt{monitoring.} Este integra a su vez toda la lógica de
acceso a la cámara y todas las clases relacionadas con el algoritmo de
conteo.

\imagen{monitoring-package}{Paquete \emph{monitoring}.}

El diagrama de clases del paquete \texttt{camera} es el siguiente:

\imagen{camera-class-diagram}{Diagrama de clases del paquete \emph{camera}.}
\newpage
El diagrama de las clases que implementan el algoritmo de conteo es el
siguiente:

\imagenAncho{algorithm-class-diagram}{Diagrama de clases del paquete \emph{algorithm}.}{1}
\newpage
En la parte del acceso a datos, se poseen dos paquetes como se ha visto
en el apartado anterior.

\imagenAncho{data-package}{Paquete \emph{data}.}{1}

El paquete \emph{model} contiene todas las clases de modelo que se
mapean con la base de datos.

\imagenAncho{model-package}{Paquete \emph{model}.}{0.8}
\newpage
\textbf{Nota:} la clase \texttt{Recording} se utiliza para agrupar a un conjunto de \texttt{Records},
pero no se almacena en la base de datos directamente (solo los Records).

Por otro lado, el paquete \texttt{source} contiene todas las clases
correspondientes a los accesos de las diferentes fuentes de datos. Su
diagrama de clases es el siguiente:

\imagenAncho{source-class-diagram}{Diagrama de clases del paquete \texttt{source}.}{1}

Para conocer a mayor detalle las funciones de cada clase se puede
consultar la documentación JavaDoc de la aplicación.

\section{Diseño procedimental}\label{diseno-procedimental}

En este apartado se recogen los detalles más relevantes respecto a la
ejecución del algoritmo de monitorización de la actividad de vuelo de
una colmena.

En el siguiente diagrama de secuencia se ha representado como es la
interacción entre los diferentes objetos que se encargan de la
inicialización de la monitorización, la obtención de las imágenes y su
posterior procesado por el algoritmo de conteo.

\begin{landscape}
\imagenAncho{algo-sequence-diagram}{Diagrama de secuencia del algoritmo.}{1.5}
\end{landscape}

\section{Diseño de interfaces}\label{diseno-de-interfaces}

En el diseño de la interfaz se ha seguido la guía de estilos de
\emph{Material Design} \citep{design:material} introducida en el Google
I/O 2014 y que se adoptó en Android a partir de la versión 5.0
(\emph{Lollipop}).

En las primeras etapas de proyecto se realizaron una serie de prototipos
en los que se plasmaron las principales funcionalidades de la
aplicación.

\imagenAncho{prototipos}{Prototipos iniciales.}{1}

Tras una serie de iteraciones, estos se fueron mejorando hasta obtener
las interfaces con las que cuenta hoy en día la \emph{app}.

\imagenAncho{features}{Diseños finales de las interfaces.}{1}

El siguiente diagrama muestra la navegabilidad por la aplicación. Esta
ha sido distribuida de acuerdo al tipo de contenido y a las tareas a
realizar sobre este.

\imagenAncho{navegation-diagram}{Diagrama de navegabilidad.}{1}

Se ha escogido la paleta de colores entre los recomendados por
\emph{Material Design}. Utilizando como principal un color en la gama de
los 500, lo que denominan un color \emph{material,} y definiendo otro
color que contraste con este para acentuar.

\imagen{palette}{Paleta de colores.}

\apendice{Documentación técnica de programación}

\section{Introducción}\label{introduccion}

En este anexo se describe la documentación técnica de programación,
incluyendo la instalación del entorno de desarrollo, la estructura de la
aplicación, su compilación, la configuración de los diferentes servicios
de integración utilizados o las baterías de test realizadas.

\section{Estructura de directorios}\label{estructura-de-directorios}

El repositorio del proyecto se distribuye de la siguiente manera:

\begin{itemize}
\tightlist
\item
  \texttt{/}: contiene los ficheros de configuración de Gradle, de los
  servicios de integración continua, el fichero README y la copia de la
  licencia.
\item
  \texttt{/app/}: módulo correspondiente a la aplicación.
\item
  \texttt{/app/src/}: código fuente de la aplicación.
\item
  \texttt{/app/src/main/}: contiene todas las clases comunes a todos los
  \emph{flavours}.
\item
  \texttt{/app/src/main/res/}: recursos de la aplicación
  (\emph{layouts}, menús, imágenes, cadenas de texto, etc.).
\item
  \texttt{/app/src/mock/}: \emph{mock flavour}, utilizado para inyectar
  componentes alternativos durante los test.
\item
  \texttt{/app/src/prod/}: \emph{prod flavour}, utilizado para inyectar
  los componentes que se utilizan en la versión de producción.
\item
  \texttt{/app/src/test/}: test unitarios.
\item
  \texttt{/app/src/testMock/}: test unitarios y de integración que
  necesitan inyectar componentes falsos.
\item
  \texttt{/app/src/androidTest/}: Android UI test.
\item
  \texttt{/docs/}: documentación del proyecto.
\item
  \texttt{/docs/img/}: imágenes utilizadas en la documentación.
\item
  \texttt{/docs/javadoc/}: documentación \emph{javadoc}.
\item
  \texttt{/docs/latex/}: documentación en formato \LaTeX.
\item
  \texttt{/docs/rst/}: documentación en formato reStructuredText.
\end{itemize}

Para saber más sobre la organización de un proyecto Android consultar
la documentación \citep{android:folders}.

\section{Manual del programador}\label{manual-del-programador-1}

El siguiente manual tiene como objetivo servir de referencia a futuros
programadores que trabajen en la aplicación. En él se explica cómo
montar el entorno de desarrollo, obtener el código fuente del proyecto,
compilarlo, ejecutarlo, testearlo y exportarlo.

\subsection{Entorno de desarrollo}\label{entorno-de-desarrollo}

Para trabajar con el proyecto se necesita tener instalados los
siguientes programas y dependencias:

\begin{itemize}
\tightlist
\item
  Java JDK 7.
\item
  Android Studio.
\item
  Git.
\item
  OpenCV.
\end{itemize}

A continuación, se indica como instalar y configurar correctamente cada
uno de ellos.

\subsubsection{Java JDK 7}\label{java-jdk-7}

El lenguaje de programación más popular para realizar aplicaciones
Android es Java. A día de hoy, Android no soporta la versión 8 de Java,
por lo que tenemos que trabajar con la versión 7. Podemos obtener esta
versión desde \citep{java:jdk7}. Se debe elegir correctamente el sistema
operativo y la arquitectura del ordenador y, posteriormente, seguir el
asistente de instalación.

\subsubsection{Android Studio}\label{android-studio}

Android Studio es el IDE oficial para el desarrollo de aplicaciones
Android. Está basado en IntelliJ IDEA de JetBrains. Proporciona soporte
para Gradle, emulador, editor de \emph{layouts}, refactorizaciones
específicas de Android, herramientas Lint para detectar problemas de
rendimiento, uso, compatibilidad de versión, etc.

Se puede obtener desde \citep{android:androidstudio}. Junto con Android
Studio se instala también el Android SDK y \emph{Android Virtual Device}
(AVD).

\imagen{android-studio}{Android Studio.}

\subsubsection{Git}\label{git}

Para hacer uso del repositorio se necesita tener instalado el gestor de
versiones Git. Este programa nos permitirá clonar el repositorio,
movernos por sus diferentes ramas, etiquetas, etc. Se puede obtener
desde \citep{git:scm}. Una vez instalado, trabajaremos con Git
Bash.

\imagen{git-clone}{Terminal Git Bash.}

\subsubsection{OpenCv}\label{opencv}

OpenCV es un paquete \emph{Open Source} de visión artificial que
contiene más de 2500 librerías de procesamiento de imágenes y visión
artificial, escritas en C/C\texttt{++} a bajo/medio nivel.

Para ejecutar OpenCV en un dispositivo Android se necesita tener
instalado la aplicación
\href{https://play.google.com/store/apps/details?id=org.opencv.engine}{OpenCV
Manager}. Sin embargo, para el desarrollo de la aplicación también
debemos instalar la versión de escritorio de OpenCV para poder ejecutar
los test de integración del algoritmo en local.

Podemos obtener OpenCV desde la página oficial \citep{opencv:web}. En
este proyecto hemos utilizado la versión 3.2.

Una vez instalada, tenemos que añadir al \emph{path} de Windows el
directorio donde se encuentran los ejecutables
(\texttt{opencv/build/java/x64} o \texttt{x86}).

\imagen{opencv-var}{Variable del sistema de OpenCV.}

En la página web oficial se puede obtener información más detallada
sobre el proceso de instalación.

\subsection{Obtención del código
fuente}\label{obtencion-del-codigo-fuente}

Para el desarrollo de la aplicación se ha utilizado un repositorio Git
hospedado en GitHub. Para obtener una copia de este hay que proceder de
la siguiente manera:

\begin{enumerate}
\def\labelenumi{\arabic{enumi}.}
\tightlist
\item
  Abrir la terminal Git Bash.
\item
  Desplazarse al directorio donde se desee copiar el repositorio
  (utilizando el comando \texttt{cd}).
\item
  Introducir el siguiente comando:\\
  \texttt{git\ clone\ https://github.com/davidmigloz/go-bees.git}
\item
  Se iniciará la descarga del repositorio, cuando finalice se dispondrá
  de una copia completa de este.
\end{enumerate}

\begin{figure}[H]
	\centering
	\includegraphics[width=0.9\textwidth]{git-clone}
	\caption{Clonar repositorio de GitHub.}\label{fig:git-clone-1}
\end{figure}

Para conocer el proceso detalladamente consultar \citep{github:clone}.

\subsection{Importar proyecto en Android
Studio}\label{importar-proyecto-en-android-studio}

Una vez obtenido el código fuente de la aplicación, tenemos que
importarlo como proyecto de Android Studio. Para ello, hay que seguir
los siguientes pasos:

\begin{enumerate}
\def\labelenumi{\arabic{enumi}.}
\tightlist
\item
  Abrir Android Studio.
\item
  Menú \texttt{File\ \textgreater{}\ Open\ldots{}}
\item
  Buscamos el directorio donde hemos clonado el repositorio.
\item
  Dentro del repositorio, seleccionamos el archivo
  \texttt{build.gradle}.
\item
  Android Studio detectará que es un proyecto Android y lo importará
  automáticamente.
\item
  Si alguna característica de las que hace uso la aplicación no se
  encuentra instalada, Android Studio mostrará un mensaje de error con
  un enlace para instalar la característica en cuestión.
\end{enumerate}

\begin{figure}[H]
	\centering
	\includegraphics[width=0.4\textwidth]{android-studio-import}
	\caption{Importar proyecto en Android Studio.}\label{fig:android-studio-import}
\end{figure}

Para conocer el proceso detalladamente consultar \citep{android:import}.

\subsection{Añadir nuevas características a la
aplicación}\label{anadir-nuevas-caracteristicas-a-la-aplicacion}

Tras importar el proyecto en Android Studio, ya estamos en disposición
de realizar modificaciones de la aplicación.

Para añadir una nueva característica siguiendo la arquitectura MVP, la
convención de paquete por característica y las metodologías TDD y
GitFlow, se deben seguir los siguientes pasos generales.

\begin{enumerate}
\def\labelenumi{\arabic{enumi}.}
\tightlist
\item
  Crear una nueva rama (\emph{feature branch}) desde la rama
  \emph{develop}:\\
  \texttt{git\ checkout\ -b\ export-data\ develop}
\item
  Crear un nuevo paquete con el nombre de la característica que se desea
  añadir (ej. \texttt{exportdata}).
\item
  Crear una interfaz (ej. \texttt{ExportDataContract.java}) que contenga
  a su vez dos interfaces. En una se deben definir las responsabilidades
  del \emph{presenter} y en la otra las de la vista. Hacer
  \emph{commit}: \\ 
  \texttt{git\ add\ -A} \\
  \texttt{git\ commit\ -m\ "Add\ export\ data\ contract\ \#x"}
\item
  Crear una clase para el \emph{presenter} (ej.\\
  \texttt{ExportDataPresenter.java}) que implemente su correspondiente
  interfaz anterior (no añadir ninguna lógica todavía). Hacer
  \emph{commit}.
\item
  Crear una clase para la vista (ej. \texttt{ExportDataFragment}) que
  descienda de \texttt{Fragment} e implemente su correspondiente
  interfaz anterior (no añadir ninguna lógica todavía). Hacer
  \emph{commit}.
\item
  Crear una clase que descienda de \texttt{AppCompatActivity} (ej. \\
  \texttt{ExportDataActivity.java}) y que enlace el modelo, el
  \emph{presenter} y la vista. Hacer \emph{commit}.
\item
  Crear un test sobre el \emph{presenter} de acuerdo a los requisitos.
  Hacer \emph{commit}.
\item
  Ejecutar el test y comprobar que no pasa.
\item
  Implementar las clases anteriores hasta conseguir que pasen el test.
  Hacer \emph{commit}.
\item
  Refactorizar el código para mejorar su calidad. Hacer \emph{commit}.
\item
  Añadir un \emph{intent} desde donde se quiera acceder a esa
  característica. Hacer \emph{commit}.
\item
  Una vez que se ha implementado correctamente la característica, se
  debe incorporar a la rama \emph{develop} y sincronizar con GitHub:\\
  \texttt{git\ checkout\ develop}\\
  \texttt{git\ merge\ -\/-no-ff\ export-data}\\
  \texttt{git\ branch\ -d\ myfeature}\\
  \texttt{git\ push\ origin\ develop}
\end{enumerate}

\subsection{Actualizar dependencias}\label{actualizar-dependencias}

Una tarea de mantenimiento común es la actualización de las dependencias
de la aplicación. Es importante tenerlas actualizadas para evitar
problemas de seguridad o funcionalidad que pudiesen tener en versiones
anteriores.

El proyecto utiliza Gradle como sistemas de construcción automática del
\emph{software}. Una de sus funcionalidades es la gestión de
dependencias. Esta permite al desarrollador definir las dependencias de
su aplicación, sus versiones y los repositorios donde se hospedan y
Gradle se encarga de descargarlas e importarlas al proyecto
automáticamente.

Las dependencias se definen en el fichero \texttt{build.gradle} del
módulo de la aplicación (\texttt{go-bees/app/build.gradle}):

\begin{figure}[H]
	\centering
	\includegraphics[width=0.8\textwidth]{dependences}
	\caption{Dependencias del proyecto.}\label{fig:dependences}
\end{figure}

Se puede observar que existen tres formas de importar las dependencias,
cada una define con un ámbito de aplicación distinto:

\begin{itemize}
\tightlist
\item
  \texttt{Compile}: estará disponible para el código de la aplicación.
\item
  \texttt{testCompile}: estará disponible en los test unitarios de la
  aplicación.
\item
  \texttt{androidTestCompile}: estará disponible en los test de
  instrumentación de la aplicación.
\end{itemize}

Para actualizar la versión de una dependencia, solamente hay que
actualizar el número de la versión que figura en la importación.
Posteriormente, se debe sincronizar Gradle (\emph{Sync Project with
Gradle Files}).

\subsection{Compilar código fuente}\label{compilar-codigo-fuente}

La compilación del proyecto se realiza mediante la tarea \texttt{build}
de Gradle. Podemos ejecutarla por línea de comandos
(\texttt{./gradlew\ build}) o mediante la interfaz de Android Studio.

\begin{figure}[H]
	\centering
	\includegraphics[width=0.5\textwidth]{gradle-build}
	\caption{Compilar proyecto.}\label{fig:gradle-build}
\end{figure}

Todos los ficheros generados durante la compilación se guardan en la
carpeta \texttt{build} del proyecto.

Para conocer el proceso detalladamente consultar
\citep{android:compilerun}.

\subsection{Ejecutar aplicación}\label{ejecutar-aplicacion}

La aplicación se puede ejecutar en un dispositivo real o en un emulador.

\subsubsection{Dispositivo real}\label{dispositivo-real}

Para ejecutar la aplicación en un dispositivo real, se debe conectar
este al equipo de desarrollo mediante un cable USB. El equipo debe tener
los \emph{drivers} del dispositivo instalado, sino no lo reconocerá.

Una vez conectado el dispositivo:

\begin{enumerate}
\def\labelenumi{\arabic{enumi}.}
\tightlist
\item
  Presionar el botón \emph{Run}.
\item
  Si el equipo reconoce el dispositivo se mostrará su nombre debajo de
  ``\emph{Connected Devices}''.
\item
  Seleccionar el dispositivo y pulsa \emph{Ok}.
\item
  Se transferirá el ejecutable de la aplicación y se instalará.
\item
  Una vez instalada, se podrá utilizar la aplicación desde el
  dispositivo.
\end{enumerate}

\subsubsection{Emulador}\label{emulador}

Un emulador (denominados \emph{Android Virtual Device} - AVD) es una
aplicación que simula el funcionamiento de un dispositivo real Android.
La creación y gestión de los emuladores se hace a través de \emph{AVD
Manager}.

Para ejecutar la aplicación en un emulador:

\begin{enumerate}
\def\labelenumi{\arabic{enumi}.}
\tightlist
\item
  Presionar el botón de Run.
\item
  Si ya se posee algún emulador instalado, se mostrará en la lista de
  \emph{Android Virtual Devices}.
\item
  Si no, presionar el botón ``\emph{Create New Virtual Device}''.
\item
  Seleccionar las características que se deseen para el emulador y pulsa
  finalizar.
\item
  Seleccionar el emulador creado y pulsar \emph{Ok}.
\item
  Se iniciará el emulador y se instalará la aplicación en él.
\item
  Una vez instalada, se podrá utilizar la aplicación desde el emulador.
\end{enumerate}

Para conocer el proceso detalladamente consultar
\citep{android:compilerun}.

\subsection{Exportar aplicación}\label{exportar-aplicaciuxf3n}

Para exportar la aplicación como un fichero \texttt{.apk}:

\begin{enumerate}
\def\labelenumi{\arabic{enumi}.}
\tightlist
\item
  Menú \emph{Build} \textgreater{} \emph{Generate APK}.
\item
  Se generará un archivo \texttt{apk} y se guardará en
  \texttt{build/output/apk}.
\end{enumerate}

Si el \texttt{apk} que se desea generar es para distribuirlo en Google
Play, este debe estar firmado. Para ello:

\begin{enumerate}
\def\labelenumi{\arabic{enumi}.}
\tightlist
\item
  Menú \emph{Build} \textgreater{} \emph{Generate Signed APK}.
\item
  Se debe seleccionar el archivo \texttt{.jks} con la clave e introducir
  su contraseña. Si no se dispone de una clave, se puede generar
  siguiendo el asistente.
\item
  Se generará un archivo \texttt{apk} firmado apto para subir al Google
  Play.
\end{enumerate}

Para conocer el proceso detalladamente consultar
\citep{android:compilerun}.

\subsection{Servicios de integración
continua}\label{servicios-de-integracion-continua}

En el repositorio se han integrado varios servicios de integración
continua para detectar fallos en el software lo antes posible,
reduciendo el impacto de estos y aumentando la calidad del código.

A continuación, se describe cada servicio y se indica cómo configurarlo.

\subsubsection{TravisCI}\label{travisci}

TravisCI es una plataforma de integración continua en la nube para
proyectos alojados en GitHub. Permite realizar una \emph{build} del
proyecto y testearla automáticamente cada vez que se realiza un
\emph{commit}, devolviendo un informe con los resultados.

Para integrar Travis en el repositorio hospedado en GitHub se debe crear
una cuenta en su página web y dar permisos de acceso al repositorio. Una
vez asociado el servicio, este se configura mediante el fichero
\texttt{travis.yml}.

Las secciones más importantes de este fichero son:

\begin{itemize}
\tightlist
\item
  \texttt{sudo}: permite definir si el usuario de la máquina virtual
  tendrá privilegios o no.
\item
  \texttt{language}: permite definir el lenguaje de programación del
  proyecto.
\item
  \texttt{jdk}: permite definir la versión del JDK.
\item
  \texttt{compiler}: permite definir el compilador.
\item
  \texttt{addons}: permite configurar \emph{plugins} instalados en
  Travis (como, por ejemplo, el \emph{plugin} de SonarQube).
\item
  \texttt{env}: permite definir variables de entorno.
\item
  \texttt{android}: permite definir las dependencias Android del
  proyecto.
\item
  \texttt{licenses}: permite aceptar las licencias de las dependencias.
\item
  \texttt{before\_install}: en esta sección se pueden definir comandos a
  ejecutar antes de los comandos de la sección \texttt{install} (por ejemplo,
  actualizar la lista de paquetes).
\item
  \texttt{install}: en esta sección se deben definir aquellos comandos
  que instalen alguna dependencia (en nuestro caso
  \texttt{python-numpy}, necesaria para compilar OpenCV).
\item
  \texttt{before\_script}: en esta sección se pueden definir comandos a
  ejecutar antes de la sección script. En nuestro caso, nos descargamos
  el código fuente de OpenCV y lo compilamos.
\item
  \texttt{script}: en esta sección se realiza la compilación del
  proyecto y se ejecutan los diferentes test unitarios y de integración.
  Además, lanza un emulador y ejecuta los test de interfaz. También
  ejecuta el motor de chequeo de SonarQube.
\item
  \texttt{after\_success}: esta sección se utiliza para recolectar datos
  generados en las secciones anteriores. En nuestro caso, se envían los
  diferentes informes de ejecución de los test a el servicio Codecov.
\item
  \texttt{cache}: permite definir los directorios a cachear entre
  ejecuciones.
\end{itemize}

Los \emph{log} de ejecución de Travis son accesibles desde
\citep{travis:gobees}.

\imagen{travis}{TravisCI.}

Para saber más, acceder a su documentación \citep{travis:doc}.

\subsubsection{Codecov}\label{codecov}

Codecov es una herramienta que permite medir el porcentaje de código que
está cubierto por un test. Además, realiza representaciones visuales de
la cobertura y gráficos de su evolución.

La forma de integrarlo en el repositorio es idéntica a cómo se hizo con
Travis. Adicionalmente, hay que configurar el \emph{script} que ejecuta
Travis para que al finalizar su ejecución envíe los resultados a
Codecov.

\texttt{after\_success:\ \ bash\ \textless{}(curl\ -s\ https://codecov.io/bash)}

La configuración de Codecov se define en el archivo
\texttt{codecov.yml}.

\imagen{codecov}{Codecov.}

Para saber más, acceder a su documentación \citep{codecov:doc}.

\subsubsection{CodeClimate}\label{codeclimate}

Codeclimate es una herramienta que realiza revisiones de código
automáticamente.

La integración se realiza de forma similar a Travis. Su fichero de
configuración es \texttt{.codeclimate.yml}.

En nuestro proyecto hemos activado los siguientes motores de chequeo:
\emph{checkstyle}, \emph{fixme}, \emph{markdownlint} y \emph{pmd}.

CodeClimate utiliza el sistema de puntación GPA (\emph{Grade Point
Average}) para indicar el rendimiento general del proyecto. La nota
máxima se corresponde con un 4.0.

Los resultados de los chequeos se encuentran disponibles en
\citep{codeclimate:gobees}.

\imagen{codeclimate}{CodeClimate.}

Para saber más, acceder a su documentación \citep{codeclimate:doc}.

SonarQube es una plataforma de código abierto para la revisión continua
de la calidad de código. Permite detectar código duplicado, violaciones
de estándares, cobertura de test unitarios, \emph{bugs} potenciales,
etc.

Para integrar el servicio hay que seguir los siguientes pasos:

\begin{enumerate}
\def\labelenumi{\arabic{enumi}.}
\tightlist
\item
  Crear una cuenta en
  \href{http://www.sonarqube.com}{www.sonarqube.com}.
\item
  Generar un \emph{token} de autenticación.
\item
  Instalar el plugin de SonarQube para Gradle (\texttt{org.sonarqube}).
\item
  Configurar SonarQube en el fichero de configuración de Gradle
  (\texttt{build.gradle}).
\item
  Ejecutar la nueva tarea sonarqube de Gradle desde Travis.
\end{enumerate}

\imagen{sonarqube-config}{Configuración de SonarQube en Gradle.}

Los resultados de los análisis son accesibles desde
\citep{sonarqube:gobees}.

\imagen{sonarqube}{SonarQube.}

Para saber más, acceder a su documentación \citep{sonarqube:doc}.

\subsubsection{VersionEye}\label{versioneye}

VersionEye es una herramienta que monitoriza las dependencias del
proyecto y envía notificaciones cuando alguna de estas está
desactualizada, es vulnerable o viola la licencia del proyecto.

El servicio se integra de forma similar a Travis. No necesita fichero de
configuración.

Cuando se libera una nueva versión de alguna dependencia o se publica
alguna vulnerabilidad, VersionEye manda una notificación. Se puede
acceder a los informes desde \citep{versioneye:gobees}.

\imagen{versioneye}{VersionEye.}

Para saber más, acceder a su documentación \citep{versioneye:doc}.

\subsubsection{Read the Docs}\label{read-the-docs}

Read the Docs es un servicio de documentación continua que permite crear
y hospedar una página web generada a partir de los distintos ficheros
Markdown o reStructuredText de la documentación. Cada vez que se realiza
un \emph{commit} en el repositorio se actualiza la versión hospedada.

Se integra en el repositorio de la misma manera que Travis. Y se
configura mediante el archivo \texttt{conf.py} ubicado en
\texttt{go-bees/docs/rst}.

Actualmente, se encuentra configurado para generar una sección en la
página web por cada archivo reStructuredText que encuentre dentro del
directorio \texttt{rst}.

\imagen{readthedocs}{Página web generada con ReadTheDocs.}

Para saber más, acceder a su documentación \citep{readthedocs:doc}.

\section{Pruebas del sistema}\label{pruebas-del-sistema}

Para verificar el funcionamiento de cada uno de los módulos de la
aplicación, su integración y la interacción con estos desde la interfaz,
se han desarrollado una serie de baterías de test.

\subsection{Test unitarios}\label{test-unitarios}

Los test unitarios comprueban la funcionalidad de un único módulo
trabajando de forma aislada. Para su escritura se han utilizado las
dependencias jUnit y Mockito.

JUnit es un \emph{framework} de Java utilizado para realizar pruebas
unitarias. Mockito es un \emph{framework} de \emph{mocking} que permite
crear objetos \emph{mock} fácilmente. Estos objetos simulan parte del
comportamiento de una clase. De esta manera, podemos aislar el módulo a
testear para que los módulos de los que depende no interfieran en los
resultados del test.

Se han escrito 120 test unitarios que testean 30 clases distintas. Se
han testeado en su mayoría los \emph{presenters} que son los que poseen
la lógica de la aplicación y no tienen ninguna dependencia al
\emph{framework} de Android. Lo que permite ejecutarlos sin necesidad de
lanzar un emulador.

\imagen{unit-test}{Test unitarios.}

\subsubsection{Ejecución de los test
unitarios}\label{ejecucion-de-los-test-unitarios}

Los test unitarios se ejecutan automáticamente en Travis cada vez que se
realiza un \emph{commit} y se hace un \emph{push} a GitHub. Pero también
se pueden ejecutar en local. Para ello:

\begin{enumerate}
\def\labelenumi{\arabic{enumi}.}
\tightlist
\item
  Seleccionar el \emph{Build Variants} \texttt{mockDebug}.
\item
  Seleccionar como tipo de vista Android.
\item
  Pulsar botón derecho en el paquete \texttt{test} \textgreater{}
  \emph{Run test in go-bees.}
\item
  Se ejecutarán todos los test y se obtendrá un informe de resultados.
\end{enumerate}

\imagen{run-unit-test}{Ejecución de los test unitarios.}

\subsection{Test del algoritmo}\label{test-del-algoritmo}

Para testear el algoritmo se han escrito varios test unitarios que
prueban cada uno de sus módulos y un test de integración
(\texttt{AreaBeesCounterTest.java}) que lo testea en su totalidad contra
tres conjuntos de fotogramas etiquetados manualmente. De esta manera, se
obtiene el error que comete el algoritmo en cada caso y se compara con
unos límites prefijados. Si por alguna modificación accidental el error
supera el límite, el test falla.

\subsubsection{Ejecución del test del
algoritmo}\label{ejecucion-del-test-del-algoritmo}

El test de integración se ejecuta automáticamente en Travis junto con
los test unitarios. También puede ser ejecutado en local, pero es
imprescindible tener instalado OpenCV en el equipo. Los pasos a seguir
son:

\begin{enumerate}
\def\labelenumi{\arabic{enumi}.}
\tightlist
\item
  Seleccionar el \emph{Build Variants} \texttt{mockDebug}.
\item
  Seleccionar como tipo de vista Android.
\item
  Pulsar botón derecho en el paquete \texttt{testMock} \textgreater{}
  \emph{Run test in go-bees.}
\item
  Se ejecutará el test y se obtendrá un informe de resultados.
\end{enumerate}

\imagen{algo-test}{Ejecución del test de integración del algoritmo.}

\subsubsection{Etiquetado de nuevos conjuntos de
fotogramas}\label{etiquetado-de-nuevos-conjuntos-de-fotogramas}

Para etiquetar videos manualmente se ha desarrollado una aplicación en
Java que facilita esta labor. La aplicación va mostrando cada fotograma
y el usuario solo tiene que pinchar encima de cada abeja existente.
Finalmente, la aplicación permite exportar los datos en un archivo
\texttt{CSV} con el formato que utiliza el test del algoritmo.

Los pasos a seguir son:

\begin{enumerate}
\def\labelenumi{\arabic{enumi}.}
\tightlist
\item
  Ejecutar la aplicación (Disponible en \citep{github:extraapps}).
\item
  Abrir el directorio que posee los fotogramas.
\item
  Marcar las abejas presentes en cada fotograma con el ratón. La
  aplicación mostrará el número del fotograma y el número de abejas
  marcadas.
\item
  Al finalizar, seleccionar guardar. La aplicación exportará los datos
  en un archivo \texttt{CSV}.
\end{enumerate}

\imagen{counting_platform}{Aplicación de etiquetado de fotogramas.}

\subsubsection{Testeo de la parametrización del
algoritmo}\label{testeo-de-la-parametrizacion-del-algoritmo}

Para desarrollar el algoritmo y parametrizarlo de forma óptima, se
desarrolló una aplicación Java que permite modificar los diferentes
parámetros de cada fase en tiempo real y calcular sus tiempos de
cómputo.

Si se desea probar nuevas parametrizaciones:

\begin{enumerate}
\def\labelenumi{\arabic{enumi}.}
\tightlist
\item
  Ejecutar la aplicación (Disponible en \citep{github:extraapps}. Es
  necesario tener instalado OpenCV en el equipo).
\item
  Seleccionar un archivo de vídeo de prueba.
\item
  En la ventana izquierda se visualiza la entrada del algoritmo y a la
  derecha existe una pestaña por cada fase de este.
\item
  En cada pestaña, a parte de la salida del algoritmo para esa fase, se
  poseen una serie de controles para parametrizar el algoritmo.
\item
  En la parte inferior izquierda se muestra los fotogramas por segundo
  que se están procesando. En la parte central, el tiempo total de
  procesado. Y en la parte derecha, el tiempo parcial de la fase en
  cuestión.
\end{enumerate}

\imagen{devplatform}{Plataforma de desarrollo del algoritmo.}

\subsection{Test de interfaz}\label{test-de-interfaz}

Por último, se han desarrollado 17 test de interfaz que testean cada uno
de los requisitos de la aplicación, a excepción del requisito de
monitorización que no fue posible testearlo en un emulador (no se puede
utilizar como \emph{feed} de la cámara de un emulador un archivo de
vídeo).

Para desarrollar los test se ha utilizado Espresso, un \emph{framework}
de \emph{testing} para Android que provee una API para escribir UI test
que simulen las interacciones de usuario con la \emph{app}.

En la siguiente tabla se relaciona cada test con el requisito que
comprueba.

\begin{table}[H]
\centering
\begin{tabular}{ll}
\toprule
Test                     & Requisitos                           \\
\midrule
\texttt{AddApiaryTest.java}       & RF-1.1 Añadir colmenar.              \\
\texttt{EditApiaryTest.java}      & RF-1.2 Editar colmenar.              \\
\texttt{DeleteApiaryTest.java}    & RF-1.3 Eliminar colmenar.            \\
\texttt{ListApiariesTest.java}    & RF-1.4 Listar colmenares.            \\
\texttt{ViewApiaryTest.java}      & RF-1.5 Ver colmenar.                 \\
\texttt{AddHiveTest.java}         & RF-2.1 Añadir colmena.               \\
\texttt{EditHiveTest.java}        & RF-2.2 Editar colmena.               \\
\texttt{DeleteHiveTest.java}      & RF-2.3 Eliminar colmena.             \\
\texttt{ListHivesTest.java}       & RF-2.4 Listar colmenas.              \\
\texttt{ViewHiveTest.java}        & RF-2.5 Ver colmena.                  \\
\texttt{AddRecordingTest.java}    & RF-3.1 Añadir grabación.             \\
\texttt{DeleteRecordingTest.java} & RF-3.2 Eliminar grabación.           \\
\texttt{ListRecordingsTest.java}  & RF-3.3 Listar grabaciones.           \\
\texttt{ViewRecordingTest.java}   & RF-3.4 Ver grabación.                \\
\texttt{SettingsTest.java}        & RF-5 Configuración de la aplicación. \\
\texttt{HelpTest.java}            & RF-6 Ayuda de la aplicación.         \\
\texttt{AboutTest.java}           & RF-7 Información de la aplicación.  \\
\bottomrule
\end{tabular}
\caption{Requisitos testeados.}
\label{requisitos-test}
\end{table}

\subsubsection{Ejecución de los test de
interfaz}\label{ejecucion-de-los-test-de-interfaz}

Para ejecutar los test de interfaz es imprescindible contar con un
dispositivo físico o un emulador. Una vez conectado, se siguen los
siguientes pasos:

\begin{enumerate}
\def\labelenumi{\arabic{enumi}.}
\tightlist
\item
  Seleccionar el \emph{Build Variants} \texttt{mockDebug}.
\item
  Seleccionar como tipo de vista Android.
\item
  Pulsar botón derecho en el paquete \texttt{androidTest} \textgreater{}
  \emph{Run test in go-bees.}
\item
  Se ejecutarán cada uno de los test en el dispositivo (Android Studio
  instala una aplicación adicional que instrumenta a la aplicación a
  testear).
\item
  Al finalizar, se obtiene un informe con los resultados.
\end{enumerate}

\apendice{Documentación de usuario}

\section{Introducción}\label{introduccion-usuario}

En este manual se detallan los requerimientos de la aplicación, cómo
instalarla en un dispositivo Android e indicaciones sobre cómo
utilizarla correctamente. Todos los procedimientos aquí descritos se
encuentran también disponibles en formato video.

\section{Requisitos de usuarios}\label{requisitos-de-usuarios}

Los requisitos mínimos para poder hacer uso de la aplicación son:

\begin{itemize}
\tightlist
\item
  Contar con un dispositivo que posea Android 4.4 (\emph{KitKat} -- API
  19) o superior.
\item
  Para utilizar la característica de monitorización de la actividad, es
  necesario tener instalada la aplicación
  \href{https://play.google.com/store/apps/details?id=org.opencv.engine}{OpenCV
  Manager}.
\item
  También se necesita contar con permiso para acceder a la cámara del
  dispositivo.
\item
  Si se desea localizar los colmenares mediante GPS, es necesario contar
  con un dispositivo que lo soporte y conceder el permiso de
  localización a la aplicación.
\item
  Para acceder a la información meteorológica se requiere conexión a
  internet.
\end{itemize}

\section{Instalación}\label{instalacion}

La instalación se puede realizar de dos maneras: a través de Google Play
o instalando directamente el ejecutable de la aplicación en nuestro
dispositivo.

\subsection{Desde Google Play}\label{desde-google-play}

Google Play es una plataforma de distribución digital de aplicaciones
móviles para los dispositivos Android. GoBees se distribuye por esta
plataforma desde su versión 1.0.

\imagen{gobees-google-play}{GoBees en Google Play.}

Video-tutorial:
\url{http://gobees.io/help/videos/instalacion-google-play}

Para instalar la aplicación debemos realizar los siguientes pasos:

\begin{enumerate}
\def\labelenumi{\arabic{enumi}.}
\tightlist
\item
  Acceder a la aplicación Google Play.
\item
  Buscar el término ``GoBees''.
\item
  Entrar en la sección correspondiente a la aplicación.
\item
  Pulsar el botón instalar.
\item
  Cuando la instalación haya finalizado, pulsar sobre el botón abrir.
\item
  La instalación habrá finalizado y la aplicación estará lista para su
  uso.
\end{enumerate}

\imagenAncho{gobees-google-play-install}{Instalación desde Google Play}{0.5}

\subsection{Desde fichero ejecutable}\label{desde-fichero-ejecutable}

La otra opción, es realizar la instalación directamente desde el fichero
ejecutable de la aplicación. Estos ficheros poseen la extensión
\texttt{.apk}. Podemos conseguir la última versión del \texttt{.apk} de
GoBees desde \citep{github:gobees_apk}.

Video-tutorial: \url{http://gobees.io/help/videos/instalacion-apk}

Una vez descargado, tenemos que seguir los siguientes pasos:

\begin{enumerate}
\def\labelenumi{\arabic{enumi}.}
\tightlist
\item
  En primer lugar, hay que permitir la instalación de ``aplicaciones con
  orígenes desconocidos''. Para ello:

  \begin{enumerate}
  \def\labelenumii{\alph{enumii}.}
  \tightlist
  \item
    Ir a ajustes del dispositivo.
  \item
    Seguridad (o Privacidad).
  \item
    Activar ``Orígenes desconocidos''.
  \end{enumerate}
\item
  Ejecutar el fichero descargado.
\item
  Pulsar el botón instalar.
\item
  Cuando la instalación haya finalizado, pulsar sobre el botón abrir.
\item
  La instalación habrá finalizado y la aplicación estará lista para su
  uso.
\end{enumerate}

\section{Manual de usuario}\label{manual-de-usuario-1}

En esta sección se describe el uso de las diferentes funcionalidades de
la aplicación.

\subsection{Generar datos de muestra}\label{generar-datos-de-muestra}

Una de las mejores maneras de aprender a utilizar una aplicación es
indagando en ella. GoBees permite generar un colmenar de prueba, de tal
manera, que podemos explorar las diferentes secciones con datos reales.

Video-tutorial:
\url{http://gobees.io/help/videos/generar-colmenar-prueba}

Para generar los datos de prueba:

\begin{enumerate}
\def\labelenumi{\arabic{enumi}.}
\tightlist
\item
  Pulsar el botón menú.
\item
  Entrar en la sección ``Ajustes''.
\item
  Seleccionar la opción ``Generar datos de muestra''.
\item
  Se generará un colmenar con tres colmenas y tres grabaciones por
  colmena.
\end{enumerate}

\imagenAncho{sample-apiary}{Colmenar de muestra.}{0.5}

\subsection{Añadir un colmenar}\label{auxf1adir-un-colmenar}

Un colmenar hace referencia al lugar o recinto donde se poseen un
conjunto de colmenas. Un colmenar posee un nombre, una localización y
unas notas.

Video-tutorial: \url{http://gobees.io/help/videos/anadir-colmenar}

Para añadir un nuevo colmenar:

\begin{enumerate}
\def\labelenumi{\arabic{enumi}.}
\tightlist
\item
  Desde la pantalla principal.
\item
  Pulsar el botón ``+''.
\item
  Definir el nombre del colmenar (obligatorio).
\item
  Definir la localización del colmenar (opcional).

  \begin{enumerate}
  \def\labelenumii{\alph{enumii}.}
  \tightlist
  \item
    Se pueden introducir manualmente las coordenadas, indicando la
    latitud y la longitud en el sistema de coordenadas geográficas.
  \item
    Alternativamente, se puede obtener la localización actual
    automáticamente pulsando el botón situado en la parte derecha (se
    necesitan permisos de localización para utilizar esta
    característica).
  \end{enumerate}
\item
  Definir unas notas sobre el colmenar (opcional). En las notas se puede
  apuntar cualquier cosa relacionada con el colmenar en general.
\item
  Pulsar el botón {$\checkmark$} para guardar el nuevo colmenar.
\end{enumerate}

\imagenAncho{add-apiary}{Añadir colmenar.}{0.5}

\subsection{Editar un colmenar}\label{editar-un-colmenar}

Los detalles de un colmenar se pueden editar en cualquier momento.

Video-tutorial: \url{http://gobees.io/help/videos/editar-colmenar}

Para editar un colmenar existente:

\begin{enumerate}
\def\labelenumi{\arabic{enumi}.}
\tightlist
\item
  Desde la pantalla principal.
\item
  Pulsar el botón de menú asociado al colmenar a editar (tres puntos
  verticales situados en la esquina superior derecha).
\item
  Seleccionar la opción de editar.
\item
  Se abrirá la pantalla de edición, donde se podrán modificar los datos
  que se deseen.
\item
  Pulsar el botón {$\checkmark$} para actualizar los datos editados.
\end{enumerate}

\subsection{Eliminar un colmenar}\label{eliminar-un-colmenar}

Al eliminar un colmenar, se eliminan también todos los datos asociados a
este (información del colmenar, colmenas, grabaciones e información
meteorológica).

Video-tutorial: \url{http://gobees.io/help/videos/eliminar-colmenar}

Para eliminar un colmenar existente:

\begin{enumerate}
\def\labelenumi{\arabic{enumi}.}
\tightlist
\item
  Desde la pantalla principal.
\item
  Pulsar el botón de menú asociado al colmenar a eliminar (tres puntos
  verticales situados en la esquina superior derecha).
\item
  Seleccionar la opción de eliminar.
\item
  El colmenar se eliminará junto con toda su información.
\end{enumerate}

\subsection{Consultar la información meteorológica de un
colmenar}\label{consultar-la-informaciuxf3n-meteoroluxf3gica-de-un-colmenar}

Para poder consultar la información meteorológica de un colmenar se
necesita que este posea una localización y que el dispositivo esté
conectado a internet. Si se cumplen estos dos requisitos, la información
meteorológica del colmenar se actualizará automáticamente de forma
periódica.

Video-tutorial:
\url{http://gobees.io/help/videos/consultar-info-meteo-colmenar}

Para consultar la información meteorológica:

\begin{enumerate}
\def\labelenumi{\arabic{enumi}.}
\tightlist
\item
  Asegurarse de que el colmenar tiene definida una localización y que se
  posee conexión a internet.
\item
  En la lista de colmenares, se puede visualizar un resumen con la
  temperatura y situación meteorológica en cada colmenar.
\item
  Si se desea consultar la información en detalle, entrar en el colmenar
  a consultar.
\item
  Desplazarse a la pestaña ``info''.
\item
  En la parte inferior podremos visualizar todos los detalles de la
  situación meteorológica actual en ese colmenar.
\end{enumerate}

Se pueden cambiar las unidades meteorológicas, para ello:

\begin{enumerate}
\def\labelenumi{\arabic{enumi}.}
\tightlist
\item
  En la pantalla principal.
\item
  Pulsar el botón menú.
\item
  Entrar en la sección ``Ajustes''.
\item
  Seleccionar ``Unidades meteorológicas''.

  \begin{enumerate}
  \def\labelenumii{\alph{enumii}.}
  \tightlist
  \item
    Sistema métrico: ºC y km/h.
  \item
    Sistema imperial: ºF y mph.
  \end{enumerate}
\end{enumerate}

\imagenAncho{meteo-info}{Información meteorológica.}{0.5
}

\subsection{Visualizar un colmenar en el
mapa}\label{visualizar-un-colmenar-en-el-mapa}

GoBees nos permite visualizar fácilmente un determinado colmenar en un
mapa utilizando nuestra aplicación de mapas favorita. De esta manera,
podemos navegar hacia él o consultar cualquier detalle cartográfico.

Video-tutorial: \url{http://gobees.io/help/videos/ver-colmenar-mapa}

Para visualizar un colmenar en el mapa:

\begin{enumerate}
\def\labelenumi{\arabic{enumi}.}
\tightlist
\item
  Entrar en el colmenar a visualizar.
\item
  Desplazarse a la pestaña ``info''.
\item
  Pulsar el botón ``mapa'' situado a la derecha de la localización del
  colmenar.
\item
  Seleccionar la aplicación con la que se desea visualizar el colmenar.
\end{enumerate}

\subsection{Añadir una colmena}\label{auxf1adir-una-colmena}

Cada colmena pertenece a un colmenar y tiene un nombre y unas notas.
Además, se puede monitorizar su actividad de vuelo, dando lugar a
grabaciones.

Video-tutorial: \url{http://gobees.io/help/videos/anadir-colmena}

Para añadir una colmena en un determinado colmenar:

\begin{enumerate}
\def\labelenumi{\arabic{enumi}.}
\tightlist
\item
  Entrar en el colmenar al que pertenecerá.
\item
  Definir el nombre de la colmena (obligatorio).
\item
  Definir unas notas sobre la colmena (opcional). En las notas se puede
  apuntar cualquier cosa relacionada con la colmena en concreto.
\item
  Pulsar el botón {$\checkmark$} para guardar la nueva colmena.
\end{enumerate}

\subsection{Editar una colmena}\label{editar-una-colmena}

Los detalles de una colmena se pueden editar en cualquier momento.

Video-tutorial: \url{http://gobees.io/help/videos/editar-colmena}

Para editar una colmena existente:

\begin{enumerate}
\def\labelenumi{\arabic{enumi}.}
\tightlist
\item
  Entrar en el colmenar al que pertenece la colmena.
\item
  Pulsar el botón de menú asociado a la colmena a editar (tres puntos
  verticales situados en la esquina superior derecha).
\item
  Seleccionar la opción de editar.
\item
  Se abrirá la pantalla de edición, donde se podrán modificar los datos
  que se deseen.
\item
  Pulsar el botón {$\checkmark$} para actualizar los datos editados.
\end{enumerate}

\subsection{Eliminar una colmena}\label{eliminar-una-colmena}

Al eliminar una colmena, se eliminan también todos los datos asociados a
esta (información de la colmena y sus grabaciones).

Video-tutorial: \url{http://gobees.io/help/videos/eliminar-colmena}

Para eliminar una colmena existente:

\begin{enumerate}
\def\labelenumi{\arabic{enumi}.}
\tightlist
\item
  Entrar en el colmenar al que pertenece la colmena.
\item
  Pulsar el botón de menú asociado a la colmena a editar (tres puntos
  verticales situados en la esquina superior derecha).
\item
  Seleccionar la opción de eliminar.
\item
  La colmena se eliminará junto con toda su información.
\end{enumerate}

\subsection{Monitorizar la actividad de vuelo de una
colmena}\label{monitorizar-la-actividad-de-vuelo-de-una-colmena}

La actividad de vuelo, junto con información previa de la colmena y
conocimiento de las condiciones locales, permite conocer al apicultor el
estado de la colmena con bastante seguridad, pudiendo determinar si esta
necesita o no una intervención.

GoBees permite monitorizar este parámetro utilizando la cámara del
\emph{smartphone}.

Video-tutorial:
\url{http://gobees.io/help/videos/monitorizacion-act-vuelo}

Para monitorizar la actividad de vuelo es necesario colocar el
\emph{smartphone} de forma fija en posición cenital a la colmena. Para
esto, se puede utilizar un trípode o un soporte similar. En la siguiente
imagen se puede ver un ejemplo de colocación:

\imagenAncho{cenital}{Colocación del \emph{smartphone} en la colmena.}{0.75}

Para mejorar los resultados de la monitorización, es recomendable que el
suelo sea de un color claro y uniforme. Si posee maleza, se puede
colocar un cartón o similar, como se muestra en la imagen.

Una vez realizado en montaje, hay que seguir los siguientes pasos dentro
de la aplicación:

\begin{enumerate}
\def\labelenumi{\arabic{enumi}.}
\tightlist
\item
  Entrar en el colmenar al que pertenece la colmena a monitorizar.
\item
  Entrar en la colmena.
\item
  Pulsar en el botón de ``monitorización'' (situado en la parte inferior
  derecha con un icono de una cámara).
\item
  Se abrirá una ventana que permite previsualizar la monitorización.
\item
  Para configurar los parámetros de la monitorización, pulsar el botón
  ``ajustes'' (situado en la parte superior derecha). Se abrirá una
  pantalla con los siguientes ajustes:

  \begin{itemize}
  \tightlist
  \item
    \textbf{Mostrar salida del algoritmo}: si no se encuentra activado
    se previsualiza la imagen proveniente de la cámara. Si se activa, se
    muestran en verde las abejas detectadas y en rojo otros objetos en
    movimiento que el algoritmo no considera abejas. Además, en la
    esquina inferior derecha se puede visualizar el número total de
    abejas contadas en cada fotograma.
  \item
    \textbf{Modificar el tamaño de las regiones}: dependiendo de la
    distancia a la que esté situada la cámara, es posible que las abejas
    se visualicen demasiado pequeñas o demasiado grandes. Con esta
    opción, se puede agrandar o disminuir su silueta.
  \item
    \textbf{Min. área abeja}: la detección de una abeja se realiza por
    área. Si el contorno en movimiento detectado posee un área dentro de
    unos límites se considera una abeja. Este parámetro configura la
    cota inferior del área. Bien ajustado, permite descartar moscas y
    mosquitos.
  \item
    \textbf{Max. área abeja}: configura la cota superior del área.
    Permite descartar la mayoría de animales que pueden habitar en el
    colmenar (avispones, roedores, lagartos o cualquier animal de mayor
    tamaño).
  \item
    \textbf{Zoom}: permite configurar el zoom de la cámara para
    encuadrar la superficie deseada.
  \item
    \textbf{Frecuencia de muestreo}: determina el intervalo de tiempo
    entre un fotograma analizado y el siguiente a analizar. Es decir, si
    se establece en 1 segundo, la aplicación captará y analizará un
    fotograma cada segundo. Cuanto mayor sea el intervalo menor será el
    consumo de batería.
  \end{itemize}
\item
  Una vez configurados los parámetros correctamente, se puede iniciar la
  monitorización pulsado el botón blanco.
\item
  Se iniciará una cuenta atrás y comenzará la monitorización. Durante
  esta, la pantalla puede estar apagada para ahorrar batería. Se puede
  aprovechar la cuenta atrás para apagarla sin influir en la
  monitorización (al manipular el móvil siempre se producen
  trepidaciones).
\item
  Cuando se desee detener la monitorización, se debe pulsar el botón
  cuadrado rojo. Una vez pulsado, se guardará la grabación y se podrá
  acceder a los detalles de esta.
\end{enumerate}

\textbf{Nota:} Si se posee alguna aplicación de ahorro de batería es imprescindible
añadir una excepción a la aplicación GoBees para que esta se pueda
ejecutar en segundo plano sin restricciones. Si no, la aplicación puede
ser cerrada durante la monitorización.

\imagen{monitoring-settings}{Ajustes de monitorización.}

\subsection{Ver los detalles de una
grabación}\label{ver-los-detalles-de-una-grabacion}

Al monitorizar una colmena se genera lo que denominamos una grabación.
Una grabación contiene los datos de actividad de vuelo de la colmena.

Video-tutorial: \url{http://gobees.io/help/videos/ver-grabacion}

Para ver los detalles de una grabación:

\begin{enumerate}
\def\labelenumi{\arabic{enumi}.}
\tightlist
\item
  Entrar en el colmenar al que pertenece la colmena monitorizada.
\item
  Entrar en la colmena.
\item
  Pulsar en la grabación sobre la que se está interesado.
\item
  Se mostrará una pantalla con dos gráficos.

  \begin{enumerate}
  \def\labelenumii{\alph{enumii}.}
  \tightlist
  \item
    El gráfico principal muestra la actividad de vuelo. En el eje de las
    Y se representa el número de abejas en vuelo y en las X los
    instantes de tiempo. Si se pulsa sobre un punto del gráfico, se
    obtiene la medida exacta en ese punto.
  \item
    El gráfico inferior muestra la información meteorológica. Existe un
    selector con tres botones: temperatura, precipitaciones y viento.
    Según se presione en uno u otro, se muestra su gráfico
    correspondiente.
  \end{enumerate}
\item
  Con ambos gráficos se puede interpretar la actividad de vuelo de la
  colmena y determinar si es una actividad normal o la colmena necesita
  una intervención.
\end{enumerate}

\imagenAncho{recording-detail}{Detalle de la grabación.}{0.5}

\subsection{Eliminar una grabación}\label{eliminar-una-grabaciuxf3n}

Al eliminar una grabación, se eliminan también todos los datos asociados
a esta.

Video-tutorial: \url{http://gobees.io/help/videos/eliminar-grabacion}

Para eliminar una grabación existente:

\begin{enumerate}
\def\labelenumi{\arabic{enumi}.}
\tightlist
\item
  Entrar en el colmenar al que pertenece la colmena monitorizada.
\item
  Entrar en la colmena.
\item
  Localizar la grabación y pulsar el botón de menú asociado a esta (tres
  puntos verticales situados en la esquina superior derecha).
\item
  Seleccionar la opción de eliminar.
\item
  La grabación se eliminará junto con toda su información.
\end{enumerate}

\subsection{Eliminar toda la información de la
aplicación}\label{eliminar-toda-la-informaciuxf3n-de-la-aplicaciuxf3n}

Si por algún motivo se desea resetear toda la información almacenada en
la aplicación, esta cuenta una opción para ello.

Video-tutorial: \url{http://gobees.io/help/videos/eliminar-datos}

Para eliminar toda la información de la aplicación:

\begin{enumerate}
\def\labelenumi{\arabic{enumi}.}
\tightlist
\item
  Pulsar el botón menú.
\item
  Entrar en la sección ``Ajustes''.
\item
  Seleccionar la opción ``Borrar todos los datos''.
\item
  Todos los datos de la aplicación serán borrados. La aplicación volverá
  al mismo estado que cuando se instaló.
\end{enumerate}

\subsection{Consultar la información sobre la
aplicación}\label{consultar-la-informaciuxf3n-sobre-la-aplicaciuxf3n}

Para conocer la versión instalada de la aplicación, los cambios
introducidos en las diferentes versiones, la licencia o el autor de esta
hay que acceder a la sección ``Acerca de GoBees''.

Video-tutorial: \url{http://gobees.io/help/videos/acerca-gobees}

Para acceder a la sección ``Acerca de GoBees'':

\begin{enumerate}
\def\labelenumi{\arabic{enumi}.}
\tightlist
\item
  Pulsar el botón menú.
\item
  Entrar en la sección ``Acerca de GoBees''.
\item
  En ella se puede visualizar la versión de la aplicación, el autor y
  las bibliotecas utilizadas para su desarrollo.
\item
  Si se presiona el botón ``Website'' se accede a la página web de
  GoBees.
\item
  Si se presiona el botón ``Licencia'' se visualiza una copia de la
  licencia de la aplicación.
\item
  Si se presiona el botón ``\emph{Changelog}'' se visualizan los cambios
  introducidos en cada versión.
\end{enumerate}

\imagenAncho{about-gobees}{Sobre GoBees.}{0.5}


\bibliography{bibliografiaAnexos}
\bibliographystyle{plainnat}

\newenvironment{bottompar}{\par\vspace*{\fill}}{\clearpage}

\begin{bottompar}
\begin{figure}[H]
	\centering
	\includegraphics[width=2cm]{ccby}
\end{figure}


\begin{center}
Este obra está bajo una licencia Creative Commons Reconocimiento 4.0 Internacional
(\href{https://creativecommons.org/licenses/by/4.0/}{CC-BY-4.0}).
\end{center}
\end{bottompar}

\end{document}
